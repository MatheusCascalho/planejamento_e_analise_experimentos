% Options for packages loaded elsewhere
\PassOptionsToPackage{unicode}{hyperref}
\PassOptionsToPackage{hyphens}{url}
%
\documentclass[
]{article}
\usepackage{amsmath,amssymb}
\usepackage{lmodern}
\usepackage{iftex}
\ifPDFTeX
  \usepackage[T1]{fontenc}
  \usepackage[utf8]{inputenc}
  \usepackage{textcomp} % provide euro and other symbols
\else % if luatex or xetex
  \usepackage{unicode-math}
  \defaultfontfeatures{Scale=MatchLowercase}
  \defaultfontfeatures[\rmfamily]{Ligatures=TeX,Scale=1}
\fi
% Use upquote if available, for straight quotes in verbatim environments
\IfFileExists{upquote.sty}{\usepackage{upquote}}{}
\IfFileExists{microtype.sty}{% use microtype if available
  \usepackage[]{microtype}
  \UseMicrotypeSet[protrusion]{basicmath} % disable protrusion for tt fonts
}{}
\makeatletter
\@ifundefined{KOMAClassName}{% if non-KOMA class
  \IfFileExists{parskip.sty}{%
    \usepackage{parskip}
  }{% else
    \setlength{\parindent}{0pt}
    \setlength{\parskip}{6pt plus 2pt minus 1pt}}
}{% if KOMA class
  \KOMAoptions{parskip=half}}
\makeatother
\usepackage{xcolor}
\usepackage[margin=1in]{geometry}
\usepackage{color}
\usepackage{fancyvrb}
\newcommand{\VerbBar}{|}
\newcommand{\VERB}{\Verb[commandchars=\\\{\}]}
\DefineVerbatimEnvironment{Highlighting}{Verbatim}{commandchars=\\\{\}}
% Add ',fontsize=\small' for more characters per line
\usepackage{framed}
\definecolor{shadecolor}{RGB}{248,248,248}
\newenvironment{Shaded}{\begin{snugshade}}{\end{snugshade}}
\newcommand{\AlertTok}[1]{\textcolor[rgb]{0.94,0.16,0.16}{#1}}
\newcommand{\AnnotationTok}[1]{\textcolor[rgb]{0.56,0.35,0.01}{\textbf{\textit{#1}}}}
\newcommand{\AttributeTok}[1]{\textcolor[rgb]{0.77,0.63,0.00}{#1}}
\newcommand{\BaseNTok}[1]{\textcolor[rgb]{0.00,0.00,0.81}{#1}}
\newcommand{\BuiltInTok}[1]{#1}
\newcommand{\CharTok}[1]{\textcolor[rgb]{0.31,0.60,0.02}{#1}}
\newcommand{\CommentTok}[1]{\textcolor[rgb]{0.56,0.35,0.01}{\textit{#1}}}
\newcommand{\CommentVarTok}[1]{\textcolor[rgb]{0.56,0.35,0.01}{\textbf{\textit{#1}}}}
\newcommand{\ConstantTok}[1]{\textcolor[rgb]{0.00,0.00,0.00}{#1}}
\newcommand{\ControlFlowTok}[1]{\textcolor[rgb]{0.13,0.29,0.53}{\textbf{#1}}}
\newcommand{\DataTypeTok}[1]{\textcolor[rgb]{0.13,0.29,0.53}{#1}}
\newcommand{\DecValTok}[1]{\textcolor[rgb]{0.00,0.00,0.81}{#1}}
\newcommand{\DocumentationTok}[1]{\textcolor[rgb]{0.56,0.35,0.01}{\textbf{\textit{#1}}}}
\newcommand{\ErrorTok}[1]{\textcolor[rgb]{0.64,0.00,0.00}{\textbf{#1}}}
\newcommand{\ExtensionTok}[1]{#1}
\newcommand{\FloatTok}[1]{\textcolor[rgb]{0.00,0.00,0.81}{#1}}
\newcommand{\FunctionTok}[1]{\textcolor[rgb]{0.00,0.00,0.00}{#1}}
\newcommand{\ImportTok}[1]{#1}
\newcommand{\InformationTok}[1]{\textcolor[rgb]{0.56,0.35,0.01}{\textbf{\textit{#1}}}}
\newcommand{\KeywordTok}[1]{\textcolor[rgb]{0.13,0.29,0.53}{\textbf{#1}}}
\newcommand{\NormalTok}[1]{#1}
\newcommand{\OperatorTok}[1]{\textcolor[rgb]{0.81,0.36,0.00}{\textbf{#1}}}
\newcommand{\OtherTok}[1]{\textcolor[rgb]{0.56,0.35,0.01}{#1}}
\newcommand{\PreprocessorTok}[1]{\textcolor[rgb]{0.56,0.35,0.01}{\textit{#1}}}
\newcommand{\RegionMarkerTok}[1]{#1}
\newcommand{\SpecialCharTok}[1]{\textcolor[rgb]{0.00,0.00,0.00}{#1}}
\newcommand{\SpecialStringTok}[1]{\textcolor[rgb]{0.31,0.60,0.02}{#1}}
\newcommand{\StringTok}[1]{\textcolor[rgb]{0.31,0.60,0.02}{#1}}
\newcommand{\VariableTok}[1]{\textcolor[rgb]{0.00,0.00,0.00}{#1}}
\newcommand{\VerbatimStringTok}[1]{\textcolor[rgb]{0.31,0.60,0.02}{#1}}
\newcommand{\WarningTok}[1]{\textcolor[rgb]{0.56,0.35,0.01}{\textbf{\textit{#1}}}}
\usepackage{graphicx}
\makeatletter
\def\maxwidth{\ifdim\Gin@nat@width>\linewidth\linewidth\else\Gin@nat@width\fi}
\def\maxheight{\ifdim\Gin@nat@height>\textheight\textheight\else\Gin@nat@height\fi}
\makeatother
% Scale images if necessary, so that they will not overflow the page
% margins by default, and it is still possible to overwrite the defaults
% using explicit options in \includegraphics[width, height, ...]{}
\setkeys{Gin}{width=\maxwidth,height=\maxheight,keepaspectratio}
% Set default figure placement to htbp
\makeatletter
\def\fps@figure{htbp}
\makeatother
\setlength{\emergencystretch}{3em} % prevent overfull lines
\providecommand{\tightlist}{%
  \setlength{\itemsep}{0pt}\setlength{\parskip}{0pt}}
\setcounter{secnumdepth}{-\maxdimen} % remove section numbering
\newlength{\cslhangindent}
\setlength{\cslhangindent}{1.5em}
\newlength{\csllabelwidth}
\setlength{\csllabelwidth}{3em}
\newlength{\cslentryspacingunit} % times entry-spacing
\setlength{\cslentryspacingunit}{\parskip}
\newenvironment{CSLReferences}[2] % #1 hanging-ident, #2 entry spacing
 {% don't indent paragraphs
  \setlength{\parindent}{0pt}
  % turn on hanging indent if param 1 is 1
  \ifodd #1
  \let\oldpar\par
  \def\par{\hangindent=\cslhangindent\oldpar}
  \fi
  % set entry spacing
  \setlength{\parskip}{#2\cslentryspacingunit}
 }%
 {}
\usepackage{calc}
\newcommand{\CSLBlock}[1]{#1\hfill\break}
\newcommand{\CSLLeftMargin}[1]{\parbox[t]{\csllabelwidth}{#1}}
\newcommand{\CSLRightInline}[1]{\parbox[t]{\linewidth - \csllabelwidth}{#1}\break}
\newcommand{\CSLIndent}[1]{\hspace{\cslhangindent}#1}
\ifLuaTeX
  \usepackage{selnolig}  % disable illegal ligatures
\fi
\IfFileExists{bookmark.sty}{\usepackage{bookmark}}{\usepackage{hyperref}}
\IfFileExists{xurl.sty}{\usepackage{xurl}}{} % add URL line breaks if available
\urlstyle{same} % disable monospaced font for URLs
\hypersetup{
  pdftitle={Estudo de Caso 03: Avaliação e comparação de duas configurações de Algoritmos Evolucionários},
  pdfauthor={Joao P. L. Pinto, Thales H. de O. Gonçalves, Henrique A. Barbosa, ~date: 23 de Dezembro de 2024},
  hidelinks,
  pdfcreator={LaTeX via pandoc}}

\title{Estudo de Caso 03: Avaliação e comparação de duas configurações
de Algoritmos Evolucionários}
\author{Joao P. L. Pinto, Thales H. de O. Gonçalves, Henrique A.
Barbosa, ~date: 23 de Dezembro de 2024}
\date{}

\begin{document}
\maketitle

\begin{verbatim}
## Registered S3 method overwritten by 'GGally':
##   method from   
##   +.gg   ggplot2
\end{verbatim}

\hypertarget{descriuxe7uxe3o-do-problema}{%
\subsection{Descrição do Problema}\label{descriuxe7uxe3o-do-problema}}

Em diversas situações, como no design de asas de aeronaves para
maximizar a eficiência aerodinâmica {[}1{]} ou no treinamento de robôs
para executar tarefas específicas {[}2{]}, é necessário minimizar (ou
maximizar) uma função de custo de interesse. Entretanto, quando não é
possível encontrar uma solução analiticamente para seu ponto de mínimo
(máximo), recorre-se a métodos heurísticos de otimização. Entre esses
métodos destacam-se os algoritmos evolucionários {[}3{]}, que simulam
processos naturais de evolução, utilizando populações de possíveis
soluções, inicialmente geradas de forma aleatória, para evoluir
gradativamente em direção a respostas que melhor atendam ao problema em
questão.

Um dos algoritmos evolucionários mais comuns e amplamente utilizados é o
algoritmo de Evolução Diferencial {[}4{]}. Nesse método, uma população
inicial de vetores-soluções é gerada aleatoriamente, e cada indivíduo da
população é avaliado com base na função de custo a ser otimizada.
Durante várias iterações, ou épocas, os indivíduos passam por processos
de mutação e cruzamento. A mutação introduz diversidade ao alterar
características de forma aleatória, enquanto o cruzamento combina as
melhores soluções, promovendo o desenvolvimento da população para um
melhor resultado. Ao final do processo, o indivíduo com o melhor
desempenho, ou ``fit'', é considerado a solução do problema.

Esse algoritmo, no entanto, apresenta múltiplos métodos de recombinação
e mutação, assim como diversos hiperparâmetros, que influenciam
diretamente seu funcionamento, podendo resultar em variações
significativas de desempenho dependendo das configurações escolhidas.
Neste trabalho, busca-se comparar o desempenho de duas configurações
distintas aplicadas a dois algoritmos de Evolução Diferencial, avaliando
a qualidade das soluções encontradas de forma independente do número de
dimensões do problema. A tabela a seguir apresenta os métodos de
recombinação e cruzamento utilizados em cada configuração do algoritmo,
assim como seus respectivos hiperparâmetros.

\[
\begin{table}[]
\begin{tabular}{ccc}
Configuração & Método de Recombinação                                                                                 & Método de Mutação                                                          \\
1            & Blend Alpha Beta Recombination for DE \cite[herrera2003taxonomy] com $\alpha=0$ e $\beta=0$                                 & Em indivíduos aleatórios, com fator de escala do vetor de diferenças $f=4$ \\
2            & Recombinação Exponencial \cite[price2006differential] com probabilidade de cada ponto do vetor ser um ponto de corte $cr=0,6$ & Nos melhores indivíduos, com fator de escala do vetor de diferenças $f=2$ 
\end{tabular}
\end{table}
\]

A função de custo que deseja-se minimizar no contexto desse trabalho é a
função Rosembrock para os limites \(-5\le x_i \le 10\), \(i=1,...,dim\)
para um problema de \(dim \in [2,150]\) dimensões. O tamanho da
população e o número máximo de iterações do algoritmo são dependentes de
\(dim\), sendo esses valores \(5*dim\) e \(100*dim\), respectivamente.
Os parâmetros experimentais dados para este estudo são:

\begin{itemize}
\tightlist
\item
  Mínima diferença de importância prática (padronizada): \(d^*=0.5\)
\item
  Significância desejada: \(\alpha=0.05\)
\item
  Potência mínima desejada: \(1-\beta=0.8\)
\end{itemize}

Para começar o código, primeiro definimos funções capazes de gerar os
problemas e definir as configurações a serem utilizadas:

\begin{Shaded}
\begin{Highlighting}[]
\NormalTok{dimensao }\OtherTok{\textless{}{-}} \DecValTok{10}
\NormalTok{fn }\OtherTok{\textless{}{-}} \ControlFlowTok{function}\NormalTok{(X)\{}
  \ControlFlowTok{if}\NormalTok{(}\SpecialCharTok{!}\FunctionTok{is.matrix}\NormalTok{(X)) X }\OtherTok{\textless{}{-}} \FunctionTok{matrix}\NormalTok{(X, }\AttributeTok{nrow =} \DecValTok{1}\NormalTok{) }\CommentTok{\# \textless{}{-} if a single vector is passed as X}
\NormalTok{  Y }\OtherTok{\textless{}{-}} \FunctionTok{apply}\NormalTok{(X, }\AttributeTok{MARGIN =} \DecValTok{1}\NormalTok{,}
             \AttributeTok{FUN =}\NormalTok{ smoof}\SpecialCharTok{::}\FunctionTok{makeRosenbrockFunction}\NormalTok{(}\AttributeTok{dimensions =}\NormalTok{ dimensao))}
  \FunctionTok{return}\NormalTok{(Y)}
\NormalTok{\}}
\CommentTok{\# testing the function on a matrix composed of 2 points}
\NormalTok{X }\OtherTok{\textless{}{-}} \FunctionTok{matrix}\NormalTok{(}\FunctionTok{runif}\NormalTok{(}\DecValTok{20}\NormalTok{), }\AttributeTok{nrow =} \DecValTok{2}\NormalTok{) }\CommentTok{\# runif gera uma distribuição uniforme aleatória}
\FunctionTok{fn}\NormalTok{(X)}
\end{Highlighting}
\end{Shaded}

\begin{verbatim}
## [1] 100.3285 154.3380
\end{verbatim}

\begin{Shaded}
\begin{Highlighting}[]
\FunctionTok{suppressPackageStartupMessages}\NormalTok{(}\FunctionTok{library}\NormalTok{(ExpDE))}


\NormalTok{configuracoes }\OtherTok{\textless{}{-}} \FunctionTok{list}\NormalTok{(}
  \FunctionTok{list}\NormalTok{(}\AttributeTok{name =} \StringTok{"Config 1"}\NormalTok{, }\AttributeTok{recpars =} \FunctionTok{list}\NormalTok{(}\AttributeTok{name =} \StringTok{"recombination\_blxAlphaBeta"}\NormalTok{, }\AttributeTok{alpha =} \DecValTok{0}\NormalTok{, }\AttributeTok{beta =} \DecValTok{0}\NormalTok{), }\AttributeTok{mutpars =} \FunctionTok{list}\NormalTok{(}\AttributeTok{name =} \StringTok{"mutation\_rand"}\NormalTok{, }\AttributeTok{f =} \DecValTok{4}\NormalTok{)),}
  \FunctionTok{list}\NormalTok{(}\AttributeTok{name =} \StringTok{"Config 2"}\NormalTok{, }\AttributeTok{recpars =} \FunctionTok{list}\NormalTok{(}\AttributeTok{name =} \StringTok{"recombination\_exp"}\NormalTok{, }\AttributeTok{cr =} \FloatTok{0.6}\NormalTok{), }\AttributeTok{mutpars =} \FunctionTok{list}\NormalTok{(}\AttributeTok{name =} \StringTok{"mutation\_best"}\NormalTok{, }\AttributeTok{f =} \DecValTok{2}\NormalTok{))}
\NormalTok{)}
\end{Highlighting}
\end{Shaded}

\hypertarget{design-dos-experimentos}{%
\section{Design dos Experimentos}\label{design-dos-experimentos}}

Conforme descrito na seção anterior, este estudo tem como objetivo
comparar o desempenho de duas configurações distintas do algoritmo de
evolução diferencial no contexto de minimização da função Rosenbrock,
considerando problemas com dimensões variando de 2 a 150. Para isso,
busca-se:

\begin{itemize}
\tightlist
\item
  Verificar se existe uma diferença estatisticamente significativa no
  desempenho médio entre as duas configurações avaliadas; e
\item
  Identificar qual das configurações apresenta o melhor desempenho, caso
  uma diferença seja constatada.
\end{itemize}

A hipótese nula estabelecida neste estudo é que os dois níveis
comparados não apresentam diferenças significativas na média de seus
desempenhos. Essa hipótese será rejeitada caso seja identificada alguma
diferença estatisticamente significativa. No entanto, o desempenho dos
algoritmos varia em função da dimensão do problema, um efeito indesejado
para a comparação dos níveis. Essa variabilidade dificulta assumir que
as condições experimentais sejam homogêneas, tornando inadequado o uso
do teste ANOVA simples para verificar diferenças na média entre os
níveis comparados.

Para isolar o efeito da dimensão do problema e tornar o experimento mais
robusto e generalizável, emprega-se uma estratégia de blocagem com base
no número de dimensões do problema. Nesse contexto, cada bloco
corresponde a uma dimensão dentro do intervalo {[}2,150{]}. Formalmente,
o modelo estatístico é definido como:

\[y_{ij}=\mu+\tau_i+\beta_j+\epsilon_{ij}\begin{cases}i=1,...,a &\\j=1,...,b\end{cases}\]
onde \(y_{ij}\) é uma observação da amostra, \(\mu\) é a média da
amostra, \(\tau_i\) o tamanho do efeito devido aos níveis, \(\beta_j\) o
efeito dos blocos, \(\epsilon_{ij}\) o resíduo, e \(a\) e \(b\) o número
de níveis e o número de blocos, respectivamente. No teste, o interesse
está exclusivamente no fator experimental referente aos níveis, de modo
que a hipótese é formulada da seguinte forma:

\[\begin{cases} H_0: \tau_{i} = 0, \forall i=1,...,a&\\H_1: \exists \tau_{i}\neq0\end{cases}\]

Para isso, é necessário determinar o número de dimensões a serem
blocadas e a quantidade de amostras por blocos para se obter os
parâmetros experimentais desejados, especificados na seção anterior.

\hypertarget{quantidade-de-dimensuxf5es}{%
\section{Quantidade de Dimensões}\label{quantidade-de-dimensuxf5es}}

Apesar de existirem 148 grupos presentes no problema, considerando cada
grupo como uma dimensão de resolução do algoritmo, não há uma
necessidade efetiva de serem testados todos os grupos de variações. Para
isso, podemos definir o poder do teste, assim como o número de grupos
presentes no experimento, de forma a descobrir o número de instâncias
necessárias.

No caso, temos que o número de grupos condiz com a quantidade de
configurações do algoritmo, ou seja, duas. Além disto, seguindo o padrão
expresso na literatura, temos que \(\beta = 0.8\). O tamanho de efeito
mínimo relevante foi definido como 0.5, uma proposta baseada no
decaimento lento da função de Rosenbrock, conseguindo detectar
dificuldades de caminhar em bacias de atração com gradientes pouco
significativos.

O código utilizado foi baseado na biblioteca CAISEr, tal como
apresentado em {[}5{]} e {[}6{]}.

\begin{Shaded}
\begin{Highlighting}[]
\FunctionTok{library}\NormalTok{(}\StringTok{"CAISEr"}\NormalTok{)}
\end{Highlighting}
\end{Shaded}

\begin{verbatim}
## 
## CAISEr version 1.0.16
## Not compatible with code developed for 0.X.Y versions
## If needed, please visit https://git.io/fjFwf for version 0.3.3
\end{verbatim}

\begin{Shaded}
\begin{Highlighting}[]
\NormalTok{dimensoes }\OtherTok{=} \FunctionTok{calc\_instances}\NormalTok{(}\DecValTok{2}\NormalTok{, }\FloatTok{0.5}\NormalTok{, }\AttributeTok{power=}\FloatTok{0.8}\NormalTok{)}\SpecialCharTok{$}\NormalTok{ninstances}
\end{Highlighting}
\end{Shaded}

\hypertarget{definiuxe7uxe3o-de-quantidade-de-amostras}{%
\section{Definição de quantidade de
amostras}\label{definiuxe7uxe3o-de-quantidade-de-amostras}}

O planejamento da quantidade de amostras de resultado para cada dimensão
é um passo importante para alcançar melhores resultados na análise de
experimentos, uma vez que com uma quantidade insuficiente de amostras
corremos o risco de observar o efeito da variação das amostras se
sobrepondo a variação das dimensões e das configuraçãoes dos algoritmos.
Conforme proposto por {[}5{]} implementamos um algoritmo iterativo para
definir o tamanho de amostras necessária para cada dimensão do problema.
O processo se deu da seguinte forma:

\begin{itemize}
\tightlist
\item
  Calculamos a quantidade de dimensões necessárias (\(n_{blocos}\)) para
  que o teste tenha poder estatístico de 80\%
\item
  Selecionamos a quantidade \(n_{blocos}\) de dimensões igualmente
  espaçadas entre sí partindo de 2 a 150 dimensões.
\end{itemize}

Depois de definir as dimensões, partimos para a definição da quantidade
de amostras por dimensão: * Definimos o desvio padrão máximo desejado
(\(se*\)): * selecionamos 10 dimensões aleatórias; * executamos, para
cada dimensão de teste e para cada configuração, 10 repetições do
algoritmo ExpDE e armazenamos os resultados. * Calculamos o desvio
padrão dos resultados e calculamos o coeficiente de variaçãod e cada
dimensão (\(CV = \frac{\sigma}{\mu}\)) e selecionamos o desvio padrão da
amostra que obteve o menor coeficiente de variação. No caso,
\(se*= 22838.19\) * Selecionamos 10 dimensões aleatórias e calculamos a
quantidade de iterações necessária para que os resultados obtidos para
essa dimensão tenham no máximo o desvio padrão especificado \(se*\)
especificado. A quantidade mínima definida foi de 2 iterações e a máxima
de 50 iterações, considerando a inviabilidade de executar mais de 50
iterações em cada dimensão e iteração. * Após obter a quantidade de
iterações de cada dimensão, selecionamos a maior quantidade de
repetições e definimos como a quantidade de todas a dimensões, com o
objetivo de garantir que no pior caso ainda teremos um bom nível de
confiança e poder do teste, e conseguentemente para todas as outras
dimensões. A quantidade de repetições definida foi de 48 repetições por
dimensão-configuração.

O algoritmo implmentado é apresentado em {[}5{]} e implementado em um
script R separado (EC3\_expecificacao\_tamanho\_amostras.R)

A quantidade de repetições pode ser, assim, definida a partir do
seguinte código:

\begin{Shaded}
\begin{Highlighting}[]
\NormalTok{sample\_iterations }\OtherTok{\textless{}{-}} \ControlFlowTok{function}\NormalTok{(instance, algo1, algo2, se\_threshold, n0, n\_max) \{}
  \CommentTok{\# Inicializando as amostras iniciais}
\NormalTok{  x1 }\OtherTok{\textless{}{-}} \FunctionTok{sample}\NormalTok{(}\FunctionTok{algo1}\NormalTok{(instance), n0, }\AttributeTok{replace =} \ConstantTok{TRUE}\NormalTok{)}
\NormalTok{  x2 }\OtherTok{\textless{}{-}} \FunctionTok{sample}\NormalTok{(}\FunctionTok{algo2}\NormalTok{(instance), n0, }\AttributeTok{replace =} \ConstantTok{TRUE}\NormalTok{)}
  
  \CommentTok{\# Número inicial de execuções}
\NormalTok{  n1 }\OtherTok{\textless{}{-}}\NormalTok{ n0}
\NormalTok{  n2 }\OtherTok{\textless{}{-}}\NormalTok{ n0}
  
  \CommentTok{\# Função para calcular erro padrão da diferença simples}
\NormalTok{  calc\_se }\OtherTok{\textless{}{-}} \ControlFlowTok{function}\NormalTok{(x1, x2) \{}
\NormalTok{    s1 }\OtherTok{\textless{}{-}} \FunctionTok{sd}\NormalTok{(x1)}
\NormalTok{    s2 }\OtherTok{\textless{}{-}} \FunctionTok{sd}\NormalTok{(x2)}
    \FunctionTok{sqrt}\NormalTok{(s1}\SpecialCharTok{\^{}}\DecValTok{2} \SpecialCharTok{/} \FunctionTok{length}\NormalTok{(x1) }\SpecialCharTok{+}\NormalTok{ s2}\SpecialCharTok{\^{}}\DecValTok{2} \SpecialCharTok{/} \FunctionTok{length}\NormalTok{(x2))}
\NormalTok{  \}}
  
  \CommentTok{\# Calcula o erro padrão inicial}
\NormalTok{  se\_current }\OtherTok{\textless{}{-}} \FunctionTok{calc\_se}\NormalTok{(x1, x2)}
  
  \CommentTok{\# Loop até atingir o limite de precisão ou o orçamento máximo}
  \ControlFlowTok{while}\NormalTok{ (se\_current }\SpecialCharTok{\textgreater{}}\NormalTok{ se\_threshold }\SpecialCharTok{\&\&}\NormalTok{ (n1 }\SpecialCharTok{+}\NormalTok{ n2) }\SpecialCharTok{\textless{}}\NormalTok{ n\_max) \{}
    \CommentTok{\# Calcula a proporção ótima de amostras}
\NormalTok{    s1 }\OtherTok{\textless{}{-}} \FunctionTok{sd}\NormalTok{(x1)}
\NormalTok{    s2 }\OtherTok{\textless{}{-}} \FunctionTok{sd}\NormalTok{(x2)}
\NormalTok{    r\_opt }\OtherTok{\textless{}{-}}\NormalTok{ s1 }\SpecialCharTok{/}\NormalTok{ s2}
    
    \CommentTok{\# Determina qual algoritmo amostrar com base na proporção}
    \ControlFlowTok{if}\NormalTok{ (n1 }\SpecialCharTok{/}\NormalTok{ n2 }\SpecialCharTok{\textless{}}\NormalTok{ r\_opt) \{}
      \CommentTok{\# Adiciona uma nova amostra para o algoritmo 1}
\NormalTok{      new\_sample }\OtherTok{\textless{}{-}} \FunctionTok{algo1}\NormalTok{(instance)}
\NormalTok{      x1 }\OtherTok{\textless{}{-}} \FunctionTok{c}\NormalTok{(x1, new\_sample)}
\NormalTok{      n1 }\OtherTok{\textless{}{-}}\NormalTok{ n1 }\SpecialCharTok{+} \DecValTok{1}
\NormalTok{    \} }\ControlFlowTok{else}\NormalTok{ \{}
      \CommentTok{\# Adiciona uma nova amostra para o algoritmo 2}
\NormalTok{      new\_sample }\OtherTok{\textless{}{-}} \FunctionTok{algo2}\NormalTok{(instance)}
\NormalTok{      x2 }\OtherTok{\textless{}{-}} \FunctionTok{c}\NormalTok{(x2, new\_sample)}
\NormalTok{      n2 }\OtherTok{\textless{}{-}}\NormalTok{ n2 }\SpecialCharTok{+} \DecValTok{1}
\NormalTok{    \}}
    
    \CommentTok{\# Atualiza o erro padrão}
\NormalTok{    se\_current }\OtherTok{\textless{}{-}} \FunctionTok{calc\_se}\NormalTok{(x1, x2)}
\NormalTok{  \}}
  
  \CommentTok{\# Retorna os resultados}
  \FunctionTok{list}\NormalTok{(}
    \AttributeTok{x1 =}\NormalTok{ x1,}
    \AttributeTok{x2 =}\NormalTok{ x2,}
    \AttributeTok{n1 =}\NormalTok{ n1,}
    \AttributeTok{n2 =}\NormalTok{ n2,}
    \AttributeTok{se =}\NormalTok{ se\_current}
\NormalTok{  )}
\NormalTok{\}}
\end{Highlighting}
\end{Shaded}

Criando wrappers para realizar as amostragens, temos o seguinte código:

\begin{Shaded}
\begin{Highlighting}[]
\NormalTok{n\_amostrar\_para\_erro }\OtherTok{=} \DecValTok{10}
\NormalTok{n\_dimensoes\_para\_erro }\OtherTok{=} \DecValTok{10}

\CommentTok{\# estimativa de amostras}
\CommentTok{\# se\_threshold, }
\NormalTok{n0}\OtherTok{=}\DecValTok{2}
\NormalTok{n\_max}\OtherTok{=}\DecValTok{50}
\NormalTok{amostras\_teste}\OtherTok{=}\DecValTok{10}
\NormalTok{se\_threshold }\OtherTok{=} \FloatTok{22838.192087}

\NormalTok{wrapper\_config1 }\OtherTok{\textless{}{-}} \ControlFlowTok{function}\NormalTok{(dimensao)\{}
\NormalTok{  selpars }\OtherTok{\textless{}{-}} \FunctionTok{list}\NormalTok{(}\AttributeTok{name =} \StringTok{"selection\_standard"}\NormalTok{)}
\NormalTok{  stopcrit }\OtherTok{\textless{}{-}} \FunctionTok{list}\NormalTok{(}\AttributeTok{names =} \StringTok{"stop\_maxeval"}\NormalTok{, }\AttributeTok{maxevals =} \DecValTok{5000} \SpecialCharTok{*}\NormalTok{ dimensao, }\AttributeTok{maxiter =} \DecValTok{100} \SpecialCharTok{*}\NormalTok{ dimensao)}
\NormalTok{  probpars }\OtherTok{\textless{}{-}} \FunctionTok{list}\NormalTok{(}\AttributeTok{name =} \StringTok{"fn"}\NormalTok{, }\AttributeTok{xmin =} \FunctionTok{rep}\NormalTok{(}\SpecialCharTok{{-}}\DecValTok{5}\NormalTok{, dimensao), }\AttributeTok{xmax =} \FunctionTok{rep}\NormalTok{(}\DecValTok{10}\NormalTok{, dimensao))}
\NormalTok{  popsize }\OtherTok{=} \DecValTok{5} \SpecialCharTok{*}\NormalTok{ dimensao}
\NormalTok{  out }\OtherTok{\textless{}{-}} \FunctionTok{ExpDE}\NormalTok{(}\AttributeTok{mutpars =}\NormalTok{ configuracoes[[}\DecValTok{1}\NormalTok{]]}\SpecialCharTok{$}\NormalTok{mutpars,}
               \AttributeTok{recpars =}\NormalTok{ configuracoes[[}\DecValTok{1}\NormalTok{]]}\SpecialCharTok{$}\NormalTok{recpars,}
               \AttributeTok{popsize =}\NormalTok{ popsize,}
               \AttributeTok{selpars =}\NormalTok{ selpars,}
               \AttributeTok{stopcrit =}\NormalTok{ stopcrit,}
               \AttributeTok{probpars =}\NormalTok{ probpars,}
               \AttributeTok{showpars =} \FunctionTok{list}\NormalTok{(}\AttributeTok{show.iters =} \StringTok{"dots"}\NormalTok{, }\AttributeTok{showevery =} \DecValTok{20}\NormalTok{))}
  \FunctionTok{return}\NormalTok{(out}\SpecialCharTok{$}\NormalTok{Fbest)}
\NormalTok{\}}
\NormalTok{wrapper\_config2 }\OtherTok{\textless{}{-}} \ControlFlowTok{function}\NormalTok{(dimensao)\{}
\NormalTok{  selpars }\OtherTok{\textless{}{-}} \FunctionTok{list}\NormalTok{(}\AttributeTok{name =} \StringTok{"selection\_standard"}\NormalTok{)}
\NormalTok{  stopcrit }\OtherTok{\textless{}{-}} \FunctionTok{list}\NormalTok{(}\AttributeTok{names =} \StringTok{"stop\_maxeval"}\NormalTok{, }\AttributeTok{maxevals =} \DecValTok{5000} \SpecialCharTok{*}\NormalTok{ dimensao, }\AttributeTok{maxiter =} \DecValTok{100} \SpecialCharTok{*}\NormalTok{ dimensao)}
\NormalTok{  probpars }\OtherTok{\textless{}{-}} \FunctionTok{list}\NormalTok{(}\AttributeTok{name =} \StringTok{"fn"}\NormalTok{, }\AttributeTok{xmin =} \FunctionTok{rep}\NormalTok{(}\SpecialCharTok{{-}}\DecValTok{5}\NormalTok{, dimensao), }\AttributeTok{xmax =} \FunctionTok{rep}\NormalTok{(}\DecValTok{10}\NormalTok{, dimensao))}
\NormalTok{  popsize }\OtherTok{=} \DecValTok{5} \SpecialCharTok{*}\NormalTok{ dimensao}
\NormalTok{  out }\OtherTok{\textless{}{-}} \FunctionTok{ExpDE}\NormalTok{(}\AttributeTok{mutpars =}\NormalTok{ configuracoes[[}\DecValTok{2}\NormalTok{]]}\SpecialCharTok{$}\NormalTok{mutpars,}
               \AttributeTok{recpars =}\NormalTok{ configuracoes[[}\DecValTok{2}\NormalTok{]]}\SpecialCharTok{$}\NormalTok{recpars,}
               \AttributeTok{popsize =}\NormalTok{ popsize,}
               \AttributeTok{selpars =}\NormalTok{ selpars,}
               \AttributeTok{stopcrit =}\NormalTok{ stopcrit,}
               \AttributeTok{probpars =}\NormalTok{ probpars,}
               \AttributeTok{showpars =} \FunctionTok{list}\NormalTok{(}\AttributeTok{show.iters =} \StringTok{"dots"}\NormalTok{, }\AttributeTok{showevery =} \DecValTok{20}\NormalTok{))}
  \FunctionTok{return}\NormalTok{(out}\SpecialCharTok{$}\NormalTok{Fbest)}
\NormalTok{\}}



\NormalTok{dimensoes\_teste }\OtherTok{=} \FunctionTok{sample}\NormalTok{(dimensoes, amostras\_teste)}
\end{Highlighting}
\end{Shaded}

Por fim, definimos o número de repetições a seguir:

\begin{Shaded}
\begin{Highlighting}[]
\CommentTok{\# \# Defina os parâmetros}
\CommentTok{\# \# dimensaoensoes, configuracoes e repeticoes devem ser definidos como vetores ou listas.}
\CommentTok{\# \# dimensoes\textless{}{-} c(10, 20)  \# Exemplos de dimensaoensões}
\CommentTok{\# repeticoes \textless{}{-} 1:n\_amostrar\_para\_erro  \# Número de repetições}
\CommentTok{\# }
\CommentTok{\# n\_blocos = calc\_instances(2, 0.5, power=0.8) \# 2: n instancias, 0.5: tamanho de efeito mínimo; power: poder do teste}
\CommentTok{\# dimensoes = seq(2,150,length.out=38)}
\CommentTok{\# dimensoes\_teste = sample(dimensoes, 10)  \# amostragem para calculo de quantidade de repetições}
\CommentTok{\# tamanho\_amostras \textless{}{-} data.frame(}
\CommentTok{\#   Dimensao = numeric(),    \# Coluna tipo numérico}
\CommentTok{\#   Iterations1 = numeric(), \# Evita fatores (para colunas de texto)}
\CommentTok{\#   Iterations2 = numeric() \# Evita fatores (para colunas de texto)}
\CommentTok{\# )}
\CommentTok{\# for (dimensao in dimensoes\_teste)\{}
\CommentTok{\#   cat("Dimensao: ", dimensao)}
\CommentTok{\#   si = sample\_iterations(}
\CommentTok{\#     instance = dimensao,}
\CommentTok{\#     algo1 = wrapper\_config1,}
\CommentTok{\#     algo2 = wrapper\_config2,}
\CommentTok{\#     se\_threshold = se\_threshold,}
\CommentTok{\#     n0 = n0,}
\CommentTok{\#     n\_max = n\_max}
\CommentTok{\#   )}
\CommentTok{\#   cat("\textbackslash{}nAmostra 1: ", si$n1,"| Amostra 2: ", si$n2, "\textbackslash{}n")}
\CommentTok{\#   tamanho\_amostras[nrow(tamanho\_amostras)+1,] \textless{}{-} list(dimensao, si$n1, si$n2)}
\CommentTok{\#   }
\CommentTok{\# \}}
\CommentTok{\#   }
\CommentTok{\# }
\CommentTok{\# \# Parâmetros adicionais para o ExpDE}
\CommentTok{\# resultados \textless{}{-} data.frame(matrix(ncol = length(configuracoes), nrow = length(dimensoes)))}
\CommentTok{\# repeticoes\_resultados \textless{}{-} data.frame(}
\CommentTok{\#   Config = character(),   \# Coluna tipo texto}
\CommentTok{\#   Dimensao = numeric(),    \# Coluna tipo numérico}
\CommentTok{\#   Repeticao = numeric(),    \# Coluna tipo double}
\CommentTok{\#   Valor = double() \# Evita fatores (para colunas de texto)}
\CommentTok{\# )}
\CommentTok{\# colnames(resultados) \textless{}{-} sapply(configuracoes, function(config) config$name)}
\CommentTok{\# rownames(resultados) \textless{}{-} dimensoes}
\CommentTok{\# repeticoes \textless{}{-} 1:48  \# Número de repetições}
\CommentTok{\# }
\CommentTok{\# \# Loop sobre dimensaoensões, configurações e repetições}
\CommentTok{\# for (dimensao in dimensoes) \{}
\CommentTok{\#   for (configuracao in configuracoes) \{}
\CommentTok{\#     resultados\_config \textless{}{-} c()  \# Vetor para armazenar os resultados das repetições}
\CommentTok{\#     i = 0}
\CommentTok{\#     for (repeticao in repeticoes) \{}
\CommentTok{\#       selpars \textless{}{-} list(name = "selection\_standard")}
\CommentTok{\#       stopcrit \textless{}{-} list(names = "stop\_maxeval", maxevals = 5000 * dimensao, maxiter = 100 * dimensao)}
\CommentTok{\#       probpars \textless{}{-} list(name = "fn", xmin = rep({-}5, dimensao), xmax = rep(10, dimensao))}
\CommentTok{\#       popsize = 5 * dimensao}
\CommentTok{\#       \# Executa o algoritmo ExpDE e armazena o resultado}
\CommentTok{\#       \# Run algorithm on problem:}
\CommentTok{\#       out \textless{}{-} ExpDE(mutpars = configuracao$mutpars,}
\CommentTok{\#                    recpars = configuracao$recpars,}
\CommentTok{\#                    popsize = popsize,}
\CommentTok{\#                    selpars = selpars,}
\CommentTok{\#                    stopcrit = stopcrit,}
\CommentTok{\#                    probpars = probpars,}
\CommentTok{\#                    showpars = list(show.iters = "dots", showevery = 20))}
\CommentTok{\#       \# Extract observation:}
\CommentTok{\#       resultados\_config \textless{}{-} c(resultados\_config, out$Fbest)  \# Supondo que "fitness" seja uma métrica retornada}
\CommentTok{\#       cat("Configuracao", configuracao$name, "| dimensão: ", dimensao, "| resultado: ", resultados\_config)}
\CommentTok{\#       repeticoes\_resultados[nrow(repeticoes\_resultados)+1,] \textless{}{-} list(configuracao$name, dimensao, i, out$Fbest)}
\CommentTok{\#       i = i+1}
\CommentTok{\#       }
\CommentTok{\#       \#print(resultados\_config)}
\CommentTok{\#     \}}
\CommentTok{\#     resultados[as.character(dimensao), configuracao$name] \textless{}{-} mean(resultados\_config)}
\CommentTok{\#   \}}
\CommentTok{\# \}}
\CommentTok{\# }
\CommentTok{\# write.csv(resultados, "resultados\_comparacao\_algoritmo.csv")}
\CommentTok{\# write.csv(repeticoes\_resultados, "resultado\_cada\_repeticao.csv")}
\CommentTok{\# }
\CommentTok{\# novo\_df \textless{}{-} data.frame(}
\CommentTok{\#   Dimensão = rep(rownames(resultados), times = ncol(resultados)),}
\CommentTok{\#   Configuração = rep(colnames(resultados), each = nrow(resultados)),}
\CommentTok{\#   Resultado = as.vector(as.matrix(resultados))}
\CommentTok{\# )}
\CommentTok{\# write.csv(novo\_df, "resultados\_comparacao\_algoritmo\_REORGANIZADO.csv")}
\end{Highlighting}
\end{Shaded}

\hypertarget{teste-de-blocagem}{%
\section{Teste de Blocagem}\label{teste-de-blocagem}}

O teste de blocagem, como explicado anteriormente, permite avaliar a
significância de parâmetros pressupondo outros como decorrentes de
variações normalmente distribuídas. Uma vez que em R a função de teste
ANOVA é capaz de separar o experimento por grupos, iremos utilizar ela
para o experimento.

\begin{Shaded}
\begin{Highlighting}[]
\NormalTok{novo\_df }\OtherTok{\textless{}{-}} \FunctionTok{read.csv}\NormalTok{(}\StringTok{\textquotesingle{}resultados\_comparacao\_algoritmo\_REORGANIZADO.csv\textquotesingle{}}\NormalTok{)}
\NormalTok{novo\_df}\SpecialCharTok{$}\NormalTok{dim\_numeric }\OtherTok{\textless{}{-}} \FunctionTok{as.numeric}\NormalTok{(novo\_df}\SpecialCharTok{$}\NormalTok{Dimensão)}
\NormalTok{novo\_df}\SpecialCharTok{$}\NormalTok{Configuração }\OtherTok{\textless{}{-}} \FunctionTok{as.factor}\NormalTok{(novo\_df}\SpecialCharTok{$}\NormalTok{Configuração)}
\NormalTok{novo\_df}\SpecialCharTok{$}\NormalTok{Dimensão }\OtherTok{\textless{}{-}} \FunctionTok{as.factor}\NormalTok{(novo\_df}\SpecialCharTok{$}\NormalTok{Dimensão)}

\NormalTok{model }\OtherTok{\textless{}{-}} \FunctionTok{aov}\NormalTok{(Resultado}\SpecialCharTok{\textasciitilde{}}\NormalTok{Configuração}\SpecialCharTok{+}\NormalTok{Dimensão,}
             \AttributeTok{data =}\NormalTok{ novo\_df)}

\FunctionTok{summary}\NormalTok{(model)}
\end{Highlighting}
\end{Shaded}

\begin{verbatim}
##              Df    Sum Sq   Mean Sq F value   Pr(>F)    
## Configuração  1 4.065e+14 4.065e+14  61.592 2.17e-09 ***
## Dimensão     37 2.993e+14 8.091e+12   1.226    0.269    
## Residuals    37 2.442e+14 6.600e+12                     
## ---
## Signif. codes:  0 '***' 0.001 '**' 0.01 '*' 0.05 '.' 0.1 ' ' 1
\end{verbatim}

\begin{Shaded}
\begin{Highlighting}[]
\FunctionTok{summary.lm}\NormalTok{(model)}\SpecialCharTok{$}\NormalTok{r.squared}
\end{Highlighting}
\end{Shaded}

\begin{verbatim}
## [1] 0.742961
\end{verbatim}

\begin{Shaded}
\begin{Highlighting}[]
\FunctionTok{library}\NormalTok{(ggplot2)}
\FunctionTok{library}\NormalTok{(dplyr)}
\end{Highlighting}
\end{Shaded}

\begin{verbatim}
## Warning: package 'dplyr' was built under R version 4.3.1
\end{verbatim}

\begin{verbatim}
## 
## Attaching package: 'dplyr'
\end{verbatim}

\begin{verbatim}
## The following objects are masked from 'package:stats':
## 
##     filter, lag
\end{verbatim}

\begin{verbatim}
## The following objects are masked from 'package:base':
## 
##     intersect, setdiff, setequal, union
\end{verbatim}

\begin{Shaded}
\begin{Highlighting}[]
\FunctionTok{arrange}\NormalTok{(novo\_df, dim\_numeric)}
\end{Highlighting}
\end{Shaded}

\begin{verbatim}
##     X Dimensão Configuração    Resultado dim_numeric
## 1   1        2     Config 1 5.856602e-03           2
## 2  39        2     Config 2 8.826468e+00           2
## 3   2        6     Config 1 1.512644e+02           6
## 4  40        6     Config 2 1.008537e+00           6
## 5   3       10     Config 1 1.412772e+04          10
## 6  41       10     Config 2 4.390087e+00          10
## 7   4       14     Config 1 9.505052e+04          14
## 8  42       14     Config 2 1.253657e+01          14
## 9   5       18     Config 1 2.075306e+05          18
## 10 43       18     Config 2 2.591185e+01          18
## 11  6       22     Config 1 3.602024e+05          22
## 12 44       22     Config 2 5.375207e+01          22
## 13  7       26     Config 1 4.745100e+05          26
## 14 45       26     Config 2 1.092262e+02          26
## 15  8       30     Config 1 6.308779e+05          30
## 16 46       30     Config 2 1.925378e+02          30
## 17  9       34     Config 1 7.854299e+05          34
## 18 47       34     Config 2 3.088857e+02          34
## 19 10       38     Config 1 9.481202e+05          38
## 20 48       38     Config 2 5.183842e+02          38
## 21 11       42     Config 1 1.132068e+06          42
## 22 49       42     Config 2 8.304513e+02          42
## 23 12       46     Config 1 1.269470e+06          46
## 24 50       46     Config 2 1.404740e+03          46
## 25 13       50     Config 1 1.502814e+06          50
## 26 51       50     Config 2 2.289142e+03          50
## 27 14       54     Config 1 1.776029e+06          54
## 28 52       54     Config 2 3.519046e+03          54
## 29 15       58     Config 1 2.393564e+06          58
## 30 53       58     Config 2 5.623412e+03          58
## 31 16       62     Config 1 3.122677e+06          62
## 32 54       62     Config 2 8.041244e+03          62
## 33 17       66     Config 1 3.697113e+06          66
## 34 55       66     Config 2 1.144929e+04          66
## 35 18       70     Config 1 4.254641e+06          70
## 36 56       70     Config 2 1.684007e+04          70
## 37 19       74     Config 1 4.447887e+06          74
## 38 57       74     Config 2 2.260578e+04          74
## 39 20       78     Config 1 4.890662e+06          78
## 40 58       78     Config 2 3.150046e+04          78
## 41 21       82     Config 1 5.377763e+06          82
## 42 59       82     Config 2 3.990159e+04          82
## 43 22       86     Config 1 5.496084e+06          86
## 44 60       86     Config 2 5.292520e+04          86
## 45 23       90     Config 1 5.877277e+06          90
## 46 61       90     Config 2 7.110297e+04          90
## 47 24       94     Config 1 6.285707e+06          94
## 48 62       94     Config 2 8.789523e+04          94
## 49 25       98     Config 1 6.770550e+06          98
## 50 63       98     Config 2 1.165097e+05          98
## 51 26      102     Config 1 6.990572e+06         102
## 52 64      102     Config 2 1.411040e+05         102
## 53 27      106     Config 1 7.288578e+06         106
## 54 65      106     Config 2 1.693954e+05         106
## 55 28      110     Config 1 7.697371e+06         110
## 56 66      110     Config 2 2.042806e+05         110
## 57 29      114     Config 1 8.146577e+06         114
## 58 67      114     Config 2 2.426795e+05         114
## 59 30      118     Config 1 8.453790e+06         118
## 60 68      118     Config 2 2.861395e+05         118
## 61 31      122     Config 1 8.884875e+06         122
## 62 69      122     Config 2 3.326667e+05         122
## 63 32      126     Config 1 9.220712e+06         126
## 64 70      126     Config 2 3.843706e+05         126
## 65 33      130     Config 1 9.541219e+06         130
## 66 71      130     Config 2 4.358276e+05         130
## 67 34      134     Config 1 1.002466e+07         134
## 68 72      134     Config 2 4.992748e+05         134
## 69 35      138     Config 1 1.022628e+07         138
## 70 73      138     Config 2 5.562384e+05         138
## 71 36      142     Config 1 1.070613e+07         142
## 72 74      142     Config 2 6.302891e+05         142
## 73 37      146     Config 1 1.119975e+07         146
## 74 75      146     Config 2 6.863231e+05         146
## 75 38      150     Config 1 1.138022e+07         150
## 76 76      150     Config 2 7.579296e+05         150
\end{verbatim}

\begin{Shaded}
\begin{Highlighting}[]
\CommentTok{\# Análise da Configuração pela Dimensão}
\NormalTok{p }\OtherTok{\textless{}{-}} \FunctionTok{ggplot}\NormalTok{(novo\_df, }\FunctionTok{aes}\NormalTok{(}\AttributeTok{x =}\NormalTok{ dim\_numeric, }
                         \AttributeTok{y =}\NormalTok{ Resultado, }
                         \AttributeTok{group =}\NormalTok{ Configuração, }
                         \AttributeTok{colour =}\NormalTok{ Configuração))}
\NormalTok{p }\SpecialCharTok{+} \FunctionTok{geom\_line}\NormalTok{(}\AttributeTok{linetype=}\DecValTok{2}\NormalTok{) }\SpecialCharTok{+} \FunctionTok{geom\_point}\NormalTok{(}\AttributeTok{size=}\DecValTok{5}\NormalTok{)}
\end{Highlighting}
\end{Shaded}

\includegraphics{case_study_03_files/figure-latex/loaddata5-1.pdf}

Podemos notar que, de fato, \ldots{}

\hypertarget{testes-post-hoc}{%
\subsection{Testes Post-Hoc}\label{testes-post-hoc}}

Apenas de forma a expandir a análise, iremos analisar o efeito ao
considerar as dimensões como os

\hypertarget{teste-anova---atualizar}{%
\subsection{Teste ANOVA - ATUALIZAR!!!}\label{teste-anova---atualizar}}

Podemos realizar o teste utilizando pacotes próprios do R da seguinte
forma:

O P valor resultante é extremamente baixo e abaixo do \(\alpha\)
definido, ou seja, podemos rejeitar a hipótese nula e concluir que de
fato existem ações melhores que outras.

\hypertarget{premissas-do-teste}{%
\subsection{Premissas do Teste}\label{premissas-do-teste}}

Dotados dos resíduos do teste, podemos conferir as premissas específicas
do próprio teste. Inicialmente, verificamos se os resíduos seguem de
fato uma normal padrão

\begin{Shaded}
\begin{Highlighting}[]
\CommentTok{\# Check normality}
\CommentTok{\#shapiro.test(model$residuals)}
\end{Highlighting}
\end{Shaded}

Podemos concluir que os resíduos de fato seguem suficientemente bem uma
distribuição normal padrão, com um certo comportamento divergente nas
caudas. Por fim, precisamos checar a igualdade de variâncias entre cada
um dos fatores, ou seja, sua homoscedasticidade.

\begin{Shaded}
\begin{Highlighting}[]
\CommentTok{\# Check homoscedasticity}
\CommentTok{\#fligner.test(novo\_df$Resultado \textasciitilde{} novo\_df$Configuração, }
\CommentTok{\#             data = novo\_df)}
\end{Highlighting}
\end{Shaded}

Novamente podemos confirmar a premissa do teste, sendo que o ANOVA com
blocos se apresenta realmente como um teste confiável para este
problema.

\hypertarget{comparauxe7uxe3o-entre-os-nuxedveis}{%
\section{Comparação Entre os
Níveis}\label{comparauxe7uxe3o-entre-os-nuxedveis}}

Após a realização do teste aleatorizado com blocos, e a rejeição de
\(H_0\), conclui-se que os níveis são significativamente diferentes
entre si, independentemente do número de dimensões do problema. Isso
indica que uma configuração de algoritmo apresenta desempenho superior
ao outro.

Para um experimento com apenas dois níveis, o ideal é executar um teste
t pareado. No entanto, como o problema em questão é um caso particular
de uma comparação múltipla com apenas dois níveis, para identificar qual
configuração oferece o melhor desempenho, decidiu-se por aplicar um
teste post-hoc de Dunnett de todos os grupos contra um. Esse teste, na
prática, equivale a um teste t quando se trata de apenas dois níveis.
Por fim, o intervalo de confiança da diferença entre os níveis é exibido
para facilitar a interpretação dos resultados.

\begin{Shaded}
\begin{Highlighting}[]
\CommentTok{\# Situation: all vs. one}
\CommentTok{\#library(multcomp)}
\DocumentationTok{\#\# comparação da configuraçao \#\#\#\#\#\#\#}
\CommentTok{\#duntest     \textless{}{-} glht(model,}
                  \CommentTok{\#  linfct = mcp(Configuração = "Dunnett"))}

\CommentTok{\#summary(duntest)}

\CommentTok{\#duntestCI   \textless{}{-} confint(duntest)}

\CommentTok{\#par(mar = c(5, 8, 4, 2), las = 1)}
\CommentTok{\#plot(duntestCI,}
    \CommentTok{\# xlab = "Mean difference")}
\CommentTok{\#}
\DocumentationTok{\#\# comparação da configuração \#\#\#\#\#\#\#}
\CommentTok{\#duntest     \textless{}{-} glht(model2,}
           \CommentTok{\#         linfct = mcp(Dimensao = "Dunnett"))}

\CommentTok{\#summary(duntest)}

\CommentTok{\#duntestCI   \textless{}{-} confint(duntest)}

\CommentTok{\#par(mar = c(5, 8, 4, 2), las = 1)}
\CommentTok{\#plot(duntestCI,}
    \CommentTok{\# xlab = "Mean difference")}
\end{Highlighting}
\end{Shaded}

É claramente perceptível que a diferença média de desempenho entre as
configurações 2 e 1 é negativa em todo o intervalo de confiança,
indicando que a configuração 2 sobressai em relação à configuração 1.
Com base nisso, a configuração 2 foi selecionada como a de melhor
desempenho.

\hypertarget{discussuxf5es-e-conclusuxf5es}{%
\section{Discussões e Conclusões}\label{discussuxf5es-e-conclusuxf5es}}

A análise comparativa do retorno médio das ações revelou insights
valiosos sobre o desempenho relativo dos ativos avaliados. O uso robusto
de métodos estatísticos, como o teste ANOVA e o teste de Tukey, permitiu
identificar diferenças significativas entre os grupos. No entanto, as
limitações observadas, como a suposição de homoscedasticidade e a falta
de variáveis macroeconômicas, ressaltam a necessidade de uma análise
mais abrangente em estudos futuros para capturar a complexidade do
mercado.

Concluímos que a metodologia aplicada não só demonstrou sua eficácia na
validação estatística das diferenças de desempenho, mas também poderá
servir como uma estrutura útil para avaliações de ativos em diversos
contextos. À medida que investidores e analistas buscam otimizar suas
decisões, a incorporação de análises mais profundas, que considerem
fatores externos, poderá enriquecer ainda mais as estratégias de
investimento. Portanto, encorajamos novas pesquisas que aprofundem essa
discussão, contribuindo para um entendimento mais sólido das dinâmicas
do mercado financeiro.

\hypertarget{bibliografia}{%
\section*{Bibliografia}\label{bibliografia}}
\addcontentsline{toc}{section}{Bibliografia}

\hypertarget{refs}{}
\begin{CSLReferences}{0}{0}
\leavevmode\vadjust pre{\hypertarget{ref-della2019application}{}}%
\CSLLeftMargin{{[}1{]} }%
\CSLRightInline{P. Della Vecchia, L. Stingo, F. Nicolosi, A. De Marco,
E. Daniele, and E. D???Amato, {``Application of game theory and
evolutionary algorithm to the regional turboprop aircraft wing
optimization,''} \emph{Evolutionary and deterministic methods for design
optimization and control with applications to industrial and societal
problems}, pp. 403--418, 2019.}

\leavevmode\vadjust pre{\hypertarget{ref-grefenstette1994evolutionary}{}}%
\CSLLeftMargin{{[}2{]} }%
\CSLRightInline{J. Grefenstette and A. Schultz, {``An evolutionary
approach to learning in robots,''} in \emph{Machine learning workshop on
robot learning}, Citeseer, 1994, pp. 659--662.}

\leavevmode\vadjust pre{\hypertarget{ref-holland1992adaptation}{}}%
\CSLLeftMargin{{[}3{]} }%
\CSLRightInline{J. H. Holland, \emph{Adaptation in natural and
artificial systems: An introductory analysis with applications to
biology, control, and artificial intelligence}. MIT press, 1992.}

\leavevmode\vadjust pre{\hypertarget{ref-price2006differential}{}}%
\CSLLeftMargin{{[}4{]} }%
\CSLRightInline{K. Price, \emph{Differential evolution: A practical
approach to global optimization}. Springer Science \& Business Media,
2006.}

\leavevmode\vadjust pre{\hypertarget{ref-campelo2019sample}{}}%
\CSLLeftMargin{{[}5{]} }%
\CSLRightInline{F. Campelo and F. Takahashi, {``Sample size estimation
for power and accuracy in the experimental comparison of algorithms,''}
\emph{Journal of Heuristics}, vol. 25, pp. 305--338, 2019.}

\leavevmode\vadjust pre{\hypertarget{ref-campelo2020sample}{}}%
\CSLLeftMargin{{[}6{]} }%
\CSLRightInline{F. Campelo and E. F. Wanner, {``Sample size calculations
for the experimental comparison of multiple algorithms on multiple
problem instances,''} \emph{Journal of Heuristics}, vol. 26, no. 6, pp.
851--883, 2020.}

\end{CSLReferences}

\end{document}
