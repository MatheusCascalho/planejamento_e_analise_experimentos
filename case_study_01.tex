% Options for packages loaded elsewhere
\PassOptionsToPackage{unicode}{hyperref}
\PassOptionsToPackage{hyphens}{url}
%
\documentclass[
]{article}
\usepackage{amsmath,amssymb}
\usepackage{iftex}
\ifPDFTeX
  \usepackage[T1]{fontenc}
  \usepackage[utf8]{inputenc}
  \usepackage{textcomp} % provide euro and other symbols
\else % if luatex or xetex
  \usepackage{unicode-math} % this also loads fontspec
  \defaultfontfeatures{Scale=MatchLowercase}
  \defaultfontfeatures[\rmfamily]{Ligatures=TeX,Scale=1}
\fi
\usepackage{lmodern}
\ifPDFTeX\else
  % xetex/luatex font selection
\fi
% Use upquote if available, for straight quotes in verbatim environments
\IfFileExists{upquote.sty}{\usepackage{upquote}}{}
\IfFileExists{microtype.sty}{% use microtype if available
  \usepackage[]{microtype}
  \UseMicrotypeSet[protrusion]{basicmath} % disable protrusion for tt fonts
}{}
\makeatletter
\@ifundefined{KOMAClassName}{% if non-KOMA class
  \IfFileExists{parskip.sty}{%
    \usepackage{parskip}
  }{% else
    \setlength{\parindent}{0pt}
    \setlength{\parskip}{6pt plus 2pt minus 1pt}}
}{% if KOMA class
  \KOMAoptions{parskip=half}}
\makeatother
\usepackage{xcolor}
\usepackage[margin=1in]{geometry}
\usepackage{color}
\usepackage{fancyvrb}
\newcommand{\VerbBar}{|}
\newcommand{\VERB}{\Verb[commandchars=\\\{\}]}
\DefineVerbatimEnvironment{Highlighting}{Verbatim}{commandchars=\\\{\}}
% Add ',fontsize=\small' for more characters per line
\usepackage{framed}
\definecolor{shadecolor}{RGB}{248,248,248}
\newenvironment{Shaded}{\begin{snugshade}}{\end{snugshade}}
\newcommand{\AlertTok}[1]{\textcolor[rgb]{0.94,0.16,0.16}{#1}}
\newcommand{\AnnotationTok}[1]{\textcolor[rgb]{0.56,0.35,0.01}{\textbf{\textit{#1}}}}
\newcommand{\AttributeTok}[1]{\textcolor[rgb]{0.13,0.29,0.53}{#1}}
\newcommand{\BaseNTok}[1]{\textcolor[rgb]{0.00,0.00,0.81}{#1}}
\newcommand{\BuiltInTok}[1]{#1}
\newcommand{\CharTok}[1]{\textcolor[rgb]{0.31,0.60,0.02}{#1}}
\newcommand{\CommentTok}[1]{\textcolor[rgb]{0.56,0.35,0.01}{\textit{#1}}}
\newcommand{\CommentVarTok}[1]{\textcolor[rgb]{0.56,0.35,0.01}{\textbf{\textit{#1}}}}
\newcommand{\ConstantTok}[1]{\textcolor[rgb]{0.56,0.35,0.01}{#1}}
\newcommand{\ControlFlowTok}[1]{\textcolor[rgb]{0.13,0.29,0.53}{\textbf{#1}}}
\newcommand{\DataTypeTok}[1]{\textcolor[rgb]{0.13,0.29,0.53}{#1}}
\newcommand{\DecValTok}[1]{\textcolor[rgb]{0.00,0.00,0.81}{#1}}
\newcommand{\DocumentationTok}[1]{\textcolor[rgb]{0.56,0.35,0.01}{\textbf{\textit{#1}}}}
\newcommand{\ErrorTok}[1]{\textcolor[rgb]{0.64,0.00,0.00}{\textbf{#1}}}
\newcommand{\ExtensionTok}[1]{#1}
\newcommand{\FloatTok}[1]{\textcolor[rgb]{0.00,0.00,0.81}{#1}}
\newcommand{\FunctionTok}[1]{\textcolor[rgb]{0.13,0.29,0.53}{\textbf{#1}}}
\newcommand{\ImportTok}[1]{#1}
\newcommand{\InformationTok}[1]{\textcolor[rgb]{0.56,0.35,0.01}{\textbf{\textit{#1}}}}
\newcommand{\KeywordTok}[1]{\textcolor[rgb]{0.13,0.29,0.53}{\textbf{#1}}}
\newcommand{\NormalTok}[1]{#1}
\newcommand{\OperatorTok}[1]{\textcolor[rgb]{0.81,0.36,0.00}{\textbf{#1}}}
\newcommand{\OtherTok}[1]{\textcolor[rgb]{0.56,0.35,0.01}{#1}}
\newcommand{\PreprocessorTok}[1]{\textcolor[rgb]{0.56,0.35,0.01}{\textit{#1}}}
\newcommand{\RegionMarkerTok}[1]{#1}
\newcommand{\SpecialCharTok}[1]{\textcolor[rgb]{0.81,0.36,0.00}{\textbf{#1}}}
\newcommand{\SpecialStringTok}[1]{\textcolor[rgb]{0.31,0.60,0.02}{#1}}
\newcommand{\StringTok}[1]{\textcolor[rgb]{0.31,0.60,0.02}{#1}}
\newcommand{\VariableTok}[1]{\textcolor[rgb]{0.00,0.00,0.00}{#1}}
\newcommand{\VerbatimStringTok}[1]{\textcolor[rgb]{0.31,0.60,0.02}{#1}}
\newcommand{\WarningTok}[1]{\textcolor[rgb]{0.56,0.35,0.01}{\textbf{\textit{#1}}}}
\usepackage{graphicx}
\makeatletter
\def\maxwidth{\ifdim\Gin@nat@width>\linewidth\linewidth\else\Gin@nat@width\fi}
\def\maxheight{\ifdim\Gin@nat@height>\textheight\textheight\else\Gin@nat@height\fi}
\makeatother
% Scale images if necessary, so that they will not overflow the page
% margins by default, and it is still possible to overwrite the defaults
% using explicit options in \includegraphics[width, height, ...]{}
\setkeys{Gin}{width=\maxwidth,height=\maxheight,keepaspectratio}
% Set default figure placement to htbp
\makeatletter
\def\fps@figure{htbp}
\makeatother
\setlength{\emergencystretch}{3em} % prevent overfull lines
\providecommand{\tightlist}{%
  \setlength{\itemsep}{0pt}\setlength{\parskip}{0pt}}
\setcounter{secnumdepth}{-\maxdimen} % remove section numbering
\ifLuaTeX
  \usepackage{selnolig}  % disable illegal ligatures
\fi
\usepackage{bookmark}
\IfFileExists{xurl.sty}{\usepackage{xurl}}{} % add URL line breaks if available
\urlstyle{same}
\hypersetup{
  pdftitle={Case Study 00: This is a template},
  pdfauthor={Author 1 and Author 2},
  hidelinks,
  pdfcreator={LaTeX via pandoc}}

\title{Case Study 00: This is a template}
\author{Author 1 and Author 2}
\date{March 00, 2015}

\begin{document}
\maketitle

\begin{verbatim}
## Installing package into '/home/enacom/R/x86_64-pc-linux-gnu-library/4.3'
## (as 'lib' is unspecified)
\end{verbatim}

\begin{verbatim}
## also installing the dependencies 'credentials', 'systemfonts', 'textshaping', 'curl', 'gert', 'gh', 'httr2', 'ragg', 'xml2', 'usethis', 'pkgdown', 'rcmdcheck', 'roxygen2', 'rversions', 'urlchecker'
\end{verbatim}

\begin{verbatim}
## Registered S3 method overwritten by 'GGally':
##   method from   
##   +.gg   ggplot2
\end{verbatim}

\subsection{Summary}\label{summary}

This document provides a template for the Case Study reports. Reports
should always start a short executive summary (call it an
\emph{Abstract} if you want) to give the reader a general idea of the
topic under investigation, the kind of analysis performed, the results
obtained, and the general recommendations of the authors.

There are (at least) two ways to render this document to \textbf{pdf} in
R: one easy and one\ldots{} well, less easy.

\subsubsection{1. The easy way}\label{the-easy-way}

Use \href{http://rstudio.com/}{RStudio} as your editor, open the
\textbf{.Rmd} file and click the \emph{Knit PDF} button at the top of
the editor.

\subsubsection{2. The slightly-less-easy
way}\label{the-slightly-less-easy-way}

If you're using any other R editor (such as the basic
\href{http://cran.r-project.org}{R} editor), you have to use the
\emph{render()} function from the \textbf{rmarkdown} package:

\begin{Shaded}
\begin{Highlighting}[]
\FunctionTok{install.packages}\NormalTok{(}\StringTok{"devtools"}\NormalTok{)                    }\CommentTok{\# you only have to install it once}
\FunctionTok{library}\NormalTok{(devtools)}
\FunctionTok{install\_github}\NormalTok{(}\StringTok{"rstudio/rmarkdown"}\NormalTok{)             }\CommentTok{\# you only have to install it once}
\FunctionTok{library}\NormalTok{(rmarkdown)}
\FunctionTok{render}\NormalTok{(}\StringTok{"report\_template.Rmd"}\NormalTok{,}\StringTok{"pdf\_document"}\NormalTok{)    }\CommentTok{\# this renders the pdf}
\end{Highlighting}
\end{Shaded}

\subsection{Experimental design}\label{experimental-design}

A section detailing the experimental setup. This is the place where you
will define your test hypotheses, e.g.:
\[\begin{cases} H_0: \mu = 10&\\H_1: \mu<10\end{cases}\]

including the reasons behind your choices of the value for \(H_0\) and
the directionality (or not) of \(H_1\).

This is also the place where you should discuss (whenever necessary)
your definitions of minimally relevant effects (\(\delta^*\)), sample
size calculations, choice of power and significance levels, and any
other relevant information about specificities in your data collection
procedures.

\subsubsection{Description of the data
collection}\label{description-of-the-data-collection}

Whenever needed, you can also include an (optional) subsection
describing the actual data collection, how it was performed, any
adaptations or unexpected events that may have occurred, etc.
Subsections like this can also be used for the sample size calculations
or any other aspect that requires a longer discussion.

\subsection{Exploratory Data Analysis}\label{exploratory-data-analysis}

The first step is to load and preprocess the data. For instance,

\begin{Shaded}
\begin{Highlighting}[]
\FunctionTok{data}\NormalTok{(mtcars)}
\NormalTok{fc}\OtherTok{\textless{}{-}}\FunctionTok{c}\NormalTok{(}\DecValTok{2}\NormalTok{,}\DecValTok{8}\SpecialCharTok{:}\DecValTok{11}\NormalTok{)}
\ControlFlowTok{for}\NormalTok{ (i }\ControlFlowTok{in} \DecValTok{1}\SpecialCharTok{:}\FunctionTok{length}\NormalTok{(fc))\{mtcars[,fc[i]]}\OtherTok{\textless{}{-}}\FunctionTok{as.factor}\NormalTok{(mtcars[,fc[i]])\}}
\FunctionTok{levels}\NormalTok{(mtcars}\SpecialCharTok{$}\NormalTok{am) }\OtherTok{\textless{}{-}} \FunctionTok{c}\NormalTok{(}\StringTok{"Automatic"}\NormalTok{,}\StringTok{"Manual"}\NormalTok{)}
\end{Highlighting}
\end{Shaded}

To get an initial feel for the relationships between the relevant
variables of your experiment it is frequently interesting to perform
some preliminary (exploratory) analysis. This is frequently referred to
as \emph{getting a feel} of your data, and can suggest procedures (such
as outlier investigation or data transformations) to experienced
experimenters.

\begin{Shaded}
\begin{Highlighting}[]
\FunctionTok{library}\NormalTok{(GGally,}\AttributeTok{quietly =}\NormalTok{ T, }\AttributeTok{warn.conflicts =}\NormalTok{ F) }\CommentTok{\# This is just me getting fancy.}
                                                \CommentTok{\# There are much simpler ways ;{-})}
\FunctionTok{ggpairs}\NormalTok{(}\AttributeTok{data=}\NormalTok{mtcars,}\AttributeTok{columns=}\FunctionTok{c}\NormalTok{(}\DecValTok{1}\NormalTok{,}\DecValTok{9}\NormalTok{),}\AttributeTok{title=}\StringTok{"MPG by transmission type"}\NormalTok{,}
        \AttributeTok{upper=}\FunctionTok{list}\NormalTok{(}\AttributeTok{combo=}\StringTok{"box"}\NormalTok{),}\AttributeTok{lower=}\FunctionTok{list}\NormalTok{(}\AttributeTok{combo=}\StringTok{"facethist"}\NormalTok{),}
        \AttributeTok{diag=}\FunctionTok{list}\NormalTok{(}\AttributeTok{continuous=}\StringTok{"densityDiag"}\NormalTok{,}\AttributeTok{discrete=}\StringTok{"barDiag"}\NormalTok{))}
\end{Highlighting}
\end{Shaded}

\begin{figure}
\centering
\includegraphics{case_study_01_files/figure-latex/pairs-1.pdf}
\caption{Exploring the effect of car transmission on mpg values}
\end{figure}

Your preliminary analysis should be described together with the plots.
In this example, two facts are immediately clear from the plots: first,
\textbf{mpg} tends to correlate well with many of the other variables,
most intensely with \textbf{drat} (positively) and \textbf{wt}
(negatively). It is also clear that many of the variables are highly
correlated (e.g., \textbf{wt} and \textbf{disp}). Second, it seems like
manual transmission models present larger values of \textbf{mpg} than
the automatic ones. In the next section a linear model will be fit to
the data in order to investigate the significance and magnitude of this
possible effect.

\subsection{Statistical Analysis}\label{statistical-analysis}

Your statistical analysis should come here. This is the place where you
should fit your statistical model, get the results of your significance
test, your effect size estimates and confidence intervals.

\begin{Shaded}
\begin{Highlighting}[]
\NormalTok{model}\OtherTok{\textless{}{-}}\FunctionTok{aov}\NormalTok{(mpg}\SpecialCharTok{\textasciitilde{}}\NormalTok{am}\SpecialCharTok{*}\NormalTok{disp,}\AttributeTok{data=}\NormalTok{mtcars)}
\FunctionTok{summary}\NormalTok{(model)}
\end{Highlighting}
\end{Shaded}

\begin{verbatim}
##             Df Sum Sq Mean Sq F value   Pr(>F)    
## am           1  405.2   405.2  47.948 1.58e-07 ***
## disp         1  420.6   420.6  49.778 1.13e-07 ***
## am:disp      1   63.7    63.7   7.537   0.0104 *  
## Residuals   28  236.6     8.4                     
## ---
## Signif. codes:  0 '***' 0.001 '**' 0.01 '*' 0.05 '.' 0.1 ' ' 1
\end{verbatim}

\subsubsection{Checking Model
Assumptions}\label{checking-model-assumptions}

The assumptions of your test should also be validated, and possible
effects of violations should also be explored.

\begin{Shaded}
\begin{Highlighting}[]
\FunctionTok{par}\NormalTok{(}\AttributeTok{mfrow=}\FunctionTok{c}\NormalTok{(}\DecValTok{2}\NormalTok{,}\DecValTok{2}\NormalTok{), }\AttributeTok{mai=}\NormalTok{.}\DecValTok{3}\SpecialCharTok{*}\FunctionTok{c}\NormalTok{(}\DecValTok{1}\NormalTok{,}\DecValTok{1}\NormalTok{,}\DecValTok{1}\NormalTok{,}\DecValTok{1}\NormalTok{))}
\FunctionTok{plot}\NormalTok{(model,}\AttributeTok{pch=}\DecValTok{16}\NormalTok{,}\AttributeTok{lty=}\DecValTok{1}\NormalTok{,}\AttributeTok{lwd=}\DecValTok{2}\NormalTok{)}
\end{Highlighting}
\end{Shaded}

\begin{figure}
\centering
\includegraphics{case_study_01_files/figure-latex/resplots-1.pdf}
\caption{Residual plots for the anova model}
\end{figure}

\subsubsection{Conclusions and
Recommendations}\label{conclusions-and-recommendations}

The discussion of your results, and the scientific/technical meaning of
the effects detected, should be placed here. Always be sure to tie your
results back to the original question of interest!

\end{document}
