% Options for packages loaded elsewhere
\PassOptionsToPackage{unicode}{hyperref}
\PassOptionsToPackage{hyphens}{url}
%
\documentclass[
]{article}
\usepackage{amsmath,amssymb}
\usepackage{iftex}
\ifPDFTeX
  \usepackage[T1]{fontenc}
  \usepackage[utf8]{inputenc}
  \usepackage{textcomp} % provide euro and other symbols
\else % if luatex or xetex
  \usepackage{unicode-math} % this also loads fontspec
  \defaultfontfeatures{Scale=MatchLowercase}
  \defaultfontfeatures[\rmfamily]{Ligatures=TeX,Scale=1}
\fi
\usepackage{lmodern}
\ifPDFTeX\else
  % xetex/luatex font selection
\fi
% Use upquote if available, for straight quotes in verbatim environments
\IfFileExists{upquote.sty}{\usepackage{upquote}}{}
\IfFileExists{microtype.sty}{% use microtype if available
  \usepackage[]{microtype}
  \UseMicrotypeSet[protrusion]{basicmath} % disable protrusion for tt fonts
}{}
\makeatletter
\@ifundefined{KOMAClassName}{% if non-KOMA class
  \IfFileExists{parskip.sty}{%
    \usepackage{parskip}
  }{% else
    \setlength{\parindent}{0pt}
    \setlength{\parskip}{6pt plus 2pt minus 1pt}}
}{% if KOMA class
  \KOMAoptions{parskip=half}}
\makeatother
\usepackage{xcolor}
\usepackage[margin=1in]{geometry}
\usepackage{color}
\usepackage{fancyvrb}
\newcommand{\VerbBar}{|}
\newcommand{\VERB}{\Verb[commandchars=\\\{\}]}
\DefineVerbatimEnvironment{Highlighting}{Verbatim}{commandchars=\\\{\}}
% Add ',fontsize=\small' for more characters per line
\usepackage{framed}
\definecolor{shadecolor}{RGB}{248,248,248}
\newenvironment{Shaded}{\begin{snugshade}}{\end{snugshade}}
\newcommand{\AlertTok}[1]{\textcolor[rgb]{0.94,0.16,0.16}{#1}}
\newcommand{\AnnotationTok}[1]{\textcolor[rgb]{0.56,0.35,0.01}{\textbf{\textit{#1}}}}
\newcommand{\AttributeTok}[1]{\textcolor[rgb]{0.13,0.29,0.53}{#1}}
\newcommand{\BaseNTok}[1]{\textcolor[rgb]{0.00,0.00,0.81}{#1}}
\newcommand{\BuiltInTok}[1]{#1}
\newcommand{\CharTok}[1]{\textcolor[rgb]{0.31,0.60,0.02}{#1}}
\newcommand{\CommentTok}[1]{\textcolor[rgb]{0.56,0.35,0.01}{\textit{#1}}}
\newcommand{\CommentVarTok}[1]{\textcolor[rgb]{0.56,0.35,0.01}{\textbf{\textit{#1}}}}
\newcommand{\ConstantTok}[1]{\textcolor[rgb]{0.56,0.35,0.01}{#1}}
\newcommand{\ControlFlowTok}[1]{\textcolor[rgb]{0.13,0.29,0.53}{\textbf{#1}}}
\newcommand{\DataTypeTok}[1]{\textcolor[rgb]{0.13,0.29,0.53}{#1}}
\newcommand{\DecValTok}[1]{\textcolor[rgb]{0.00,0.00,0.81}{#1}}
\newcommand{\DocumentationTok}[1]{\textcolor[rgb]{0.56,0.35,0.01}{\textbf{\textit{#1}}}}
\newcommand{\ErrorTok}[1]{\textcolor[rgb]{0.64,0.00,0.00}{\textbf{#1}}}
\newcommand{\ExtensionTok}[1]{#1}
\newcommand{\FloatTok}[1]{\textcolor[rgb]{0.00,0.00,0.81}{#1}}
\newcommand{\FunctionTok}[1]{\textcolor[rgb]{0.13,0.29,0.53}{\textbf{#1}}}
\newcommand{\ImportTok}[1]{#1}
\newcommand{\InformationTok}[1]{\textcolor[rgb]{0.56,0.35,0.01}{\textbf{\textit{#1}}}}
\newcommand{\KeywordTok}[1]{\textcolor[rgb]{0.13,0.29,0.53}{\textbf{#1}}}
\newcommand{\NormalTok}[1]{#1}
\newcommand{\OperatorTok}[1]{\textcolor[rgb]{0.81,0.36,0.00}{\textbf{#1}}}
\newcommand{\OtherTok}[1]{\textcolor[rgb]{0.56,0.35,0.01}{#1}}
\newcommand{\PreprocessorTok}[1]{\textcolor[rgb]{0.56,0.35,0.01}{\textit{#1}}}
\newcommand{\RegionMarkerTok}[1]{#1}
\newcommand{\SpecialCharTok}[1]{\textcolor[rgb]{0.81,0.36,0.00}{\textbf{#1}}}
\newcommand{\SpecialStringTok}[1]{\textcolor[rgb]{0.31,0.60,0.02}{#1}}
\newcommand{\StringTok}[1]{\textcolor[rgb]{0.31,0.60,0.02}{#1}}
\newcommand{\VariableTok}[1]{\textcolor[rgb]{0.00,0.00,0.00}{#1}}
\newcommand{\VerbatimStringTok}[1]{\textcolor[rgb]{0.31,0.60,0.02}{#1}}
\newcommand{\WarningTok}[1]{\textcolor[rgb]{0.56,0.35,0.01}{\textbf{\textit{#1}}}}
\usepackage{graphicx}
\makeatletter
\def\maxwidth{\ifdim\Gin@nat@width>\linewidth\linewidth\else\Gin@nat@width\fi}
\def\maxheight{\ifdim\Gin@nat@height>\textheight\textheight\else\Gin@nat@height\fi}
\makeatother
% Scale images if necessary, so that they will not overflow the page
% margins by default, and it is still possible to overwrite the defaults
% using explicit options in \includegraphics[width, height, ...]{}
\setkeys{Gin}{width=\maxwidth,height=\maxheight,keepaspectratio}
% Set default figure placement to htbp
\makeatletter
\def\fps@figure{htbp}
\makeatother
\setlength{\emergencystretch}{3em} % prevent overfull lines
\providecommand{\tightlist}{%
  \setlength{\itemsep}{0pt}\setlength{\parskip}{0pt}}
\setcounter{secnumdepth}{-\maxdimen} % remove section numbering
% definitions for citeproc citations
\NewDocumentCommand\citeproctext{}{}
\NewDocumentCommand\citeproc{mm}{%
  \begingroup\def\citeproctext{#2}\cite{#1}\endgroup}
\makeatletter
 % allow citations to break across lines
 \let\@cite@ofmt\@firstofone
 % avoid brackets around text for \cite:
 \def\@biblabel#1{}
 \def\@cite#1#2{{#1\if@tempswa , #2\fi}}
\makeatother
\newlength{\cslhangindent}
\setlength{\cslhangindent}{1.5em}
\newlength{\csllabelwidth}
\setlength{\csllabelwidth}{3em}
\newenvironment{CSLReferences}[2] % #1 hanging-indent, #2 entry-spacing
 {\begin{list}{}{%
  \setlength{\itemindent}{0pt}
  \setlength{\leftmargin}{0pt}
  \setlength{\parsep}{0pt}
  % turn on hanging indent if param 1 is 1
  \ifodd #1
   \setlength{\leftmargin}{\cslhangindent}
   \setlength{\itemindent}{-1\cslhangindent}
  \fi
  % set entry spacing
  \setlength{\itemsep}{#2\baselineskip}}}
 {\end{list}}
\usepackage{calc}
\newcommand{\CSLBlock}[1]{\hfill\break\parbox[t]{\linewidth}{\strut\ignorespaces#1\strut}}
\newcommand{\CSLLeftMargin}[1]{\parbox[t]{\csllabelwidth}{\strut#1\strut}}
\newcommand{\CSLRightInline}[1]{\parbox[t]{\linewidth - \csllabelwidth}{\strut#1\strut}}
\newcommand{\CSLIndent}[1]{\hspace{\cslhangindent}#1}
\ifLuaTeX
  \usepackage{selnolig}  % disable illegal ligatures
\fi
\usepackage{bookmark}
\IfFileExists{xurl.sty}{\usepackage{xurl}}{} % add URL line breaks if available
\urlstyle{same}
\hypersetup{
  pdftitle={Case Study 00: This is a template},
  pdfauthor={Author 1 and Author 2},
  hidelinks,
  pdfcreator={LaTeX via pandoc}}

\title{Case Study 00: This is a template}
\author{Author 1 and Author 2}
\date{March 00, 2015}

\begin{document}
\maketitle

\begin{verbatim}
## Installing package into '/home/enacom/R/x86_64-pc-linux-gnu-library/4.3'
## (as 'lib' is unspecified)
\end{verbatim}

\begin{verbatim}
## also installing the dependencies 'credentials', 'systemfonts', 'textshaping', 'curl', 'gert', 'gh', 'httr2', 'ragg', 'xml2', 'usethis', 'pkgdown', 'rcmdcheck', 'roxygen2', 'rversions', 'urlchecker'
\end{verbatim}

\begin{verbatim}
## Registered S3 method overwritten by 'GGally':
##   method from   
##   +.gg   ggplot2
\end{verbatim}

\subsection{Descrição do Problema}\label{descriuxe7uxe3o-do-problema}

O IMC é uma medida utilizada na área de saúde como um indicador simples
capaz de encontrar correlações entre o peso e altura do paciente com
doenças decorrentes de obesidade, sendo inclusive capaz de classificar
em diferentes grupos para obter maior precisão na análise em questão
{[}1{]}, {[}2{]}.

Apesar dos problemas decorrentes da simplicidade do teste {[}3{]},
{[}4{]} e métodos mais robustos tais como a bioimpedância {[}5{]}
estarem presentes, continua sendo uma medida interessante pela
facilidade e baixo custo de encontrá-la. A partir de tal importância
médica, muitas questões naturalmente surgem relacionadas a testes entre
diferentes populações.

Neste estudo, serão propostos três testes de forma a compreender
possíveis relações e diferenças presentes na amostragem realizada nos
anos de 2016 e 2017 no Departamento de Engenharia da UFMG. Os testes
propostos são:

\begin{itemize}
\tightlist
\item
  Diferença entre IMC de Homens e Mulheres;
\item
  Diferença de IMC entre os dois anos em estudo;
\item
  Diferença entre alunos de Graduação e Pós-Graduação em 2016.
\end{itemize}

Por meio destes testes, podemos obter informações relevantes capazes de
compreender diferenças baseadas em idade, sexo e ano de estudo.

\subsection{Design dos Experimentos}\label{design-dos-experimentos}

Nos experimentos aqui realizados, procuramos extrair o máximo de
informações possíveis de dados que já foram extraídos. A partir disto,
problemas estatísticos naturalmente irão surgir, sendo alguns deles:

\begin{itemize}
\tightlist
\item
  Diferentes tamanhos de amostra;
\item
  Poucos dados;
\item
  Amostra não representativa de todo o departamento.
\end{itemize}

Apesar de tais problemas existirem, será aproveitada ao máximo a
robustez existente nos experimentos.

\subsubsection{Diferença entre IMC de Homens e
Mulheres}\label{diferenuxe7a-entre-imc-de-homens-e-mulheres}

Dada a conhecida diferença na composição corporal entre homens e
mulheres {[}6{]}, torna-se relevante analisar os índices de IMC dentro
do universo em estudo. Em termos gerais, observam-se valores semelhantes
de IMC entre os sexos {[}7{]}, apesar das variações na composição
corporal. Para aprofundar essa análise, podemos testar a hipótese
alternativa de que o IMC médio masculino (\(\mu_{imcm}\)) seja superior
ao feminino (\(\mu_{imcf}\)).
\[\begin{cases} H_0: \mu_{imcf} = \mu_{imcm}&\\H_1: \mu_{imcf}<\mu_{imcm}\end{cases}\]

Estudos referentes à normalidade de ambas distribuições foram
realizados, assim como um teste de comparação das variâncias de
Fligner-Killen.

O valor de (\(\delta^*\)) utilizado neste teste foi selecionado com base
na relevância das faixas pré-definidas de IMC, sendo que para que pessoa
no meio de uma faixa precisa ter uma alteração de 2.5 em seu IMC para
alternar de faixa, sendo (\(\delta^*=2.5\)).

O valor de \(\alpha=0.05\) buscou manter o padrão encontrado na
literatura médica, assim como esperava-se encontrar um valor de
\(\beta=0.8\) para corroborar com não encontrarmos falsos negativos.

\subsubsection{Diferença de IMC entre os dois anos em
estudo;}\label{diferenuxe7a-de-imc-entre-os-dois-anos-em-estudo}

Realizamos para este teste dois experimentos, um para o sexo masculino e
outro para o feminino. Esta estratificação baseou-se no mesmo critério
da suposta diferença dos IMCs entre os sexos.

Para cada um dos testes, iremos avaliar se (\(\mu_{2016}\)) é igual a
(\(\mu_{2017}\)).

\[\begin{cases} H_0: \mu_{2016} = \mu_{2017}&\\H_1: \mu_{2016} \neq \mu_{2017}\end{cases}\]
Comentários referentes ao \(\delta^*\), \(\alpha\) e \(\beta\) são
análogos aos do primeiro experimento.

\subsubsection{Diferença entre alunos de Graduação e Pós-Graduação em
2016;}\label{diferenuxe7a-entre-alunos-de-graduauxe7uxe3o-e-puxf3s-graduauxe7uxe3o-em-2016}

Realizamos para este experimento um teste entre a diferença da média do
IMC entre alunos de graduação (\(\mu_{grad}\)) e pós graduação
(\(\mu_{pos}\)) de forma a testar se a idade é um fator que influencia a
saúde dos alunos do departamento. Um estudo análogo, realizado na
University of South Alabama, apresentou um IMC médio levemente superior
de alunos da pós-graduação em enfermagem, em relação aos alunos da
graduação do mesmo curso {[}8{]}.

\[\begin{cases} H_0: \mu_{grad} = \mu_{pos}&\\H_1: \mu_{grad} \neq \mu_{pos}\end{cases}\]
Comentários referentes ao \(\delta^*\), \(\alpha\) e \(\beta\) são
análogos aos do primeiro experimento.

\subsection{Experimentos}\label{experimentos}

\subsubsection{Diferença entre IMC de Homens e
Mulheres}\label{diferenuxe7a-entre-imc-de-homens-e-mulheres-1}

O primeiro passo é carregar os dados e calcular o IMC das amostras.

\begin{Shaded}
\begin{Highlighting}[]
\NormalTok{imc2016 }\OtherTok{\textless{}{-}} \FunctionTok{read.csv}\NormalTok{(}\StringTok{\textquotesingle{}imc\_20162.csv\textquotesingle{}}\NormalTok{)}
\NormalTok{imc2017 }\OtherTok{\textless{}{-}} \FunctionTok{read.csv}\NormalTok{(}\StringTok{\textquotesingle{}CS01\_20172.csv\textquotesingle{}}\NormalTok{, }\AttributeTok{sep =} \StringTok{\textquotesingle{};\textquotesingle{}}\NormalTok{)}

\CommentTok{\# Filtragem de dados}
\NormalTok{imc2016 }\OtherTok{\textless{}{-}}\NormalTok{ imc2016[imc2016}\SpecialCharTok{$}\NormalTok{Course }\SpecialCharTok{==} \StringTok{\textquotesingle{}PPGEE\textquotesingle{}}\NormalTok{,]}

\CommentTok{\# Padronização do dataframe}
\FunctionTok{colnames}\NormalTok{(imc2016)[}\FunctionTok{colnames}\NormalTok{(imc2016)}\SpecialCharTok{==}\StringTok{\textquotesingle{}Gender\textquotesingle{}}\NormalTok{] }\OtherTok{\textless{}{-}} \StringTok{\textquotesingle{}Sex\textquotesingle{}}
\FunctionTok{colnames}\NormalTok{(imc2016)[}\FunctionTok{colnames}\NormalTok{(imc2016)}\SpecialCharTok{==}\StringTok{\textquotesingle{}Height.m\textquotesingle{}}\NormalTok{] }\OtherTok{\textless{}{-}} \StringTok{\textquotesingle{}height.m\textquotesingle{}}
\FunctionTok{colnames}\NormalTok{(imc2016)[}\FunctionTok{colnames}\NormalTok{(imc2016)}\SpecialCharTok{==}\StringTok{\textquotesingle{}Weight.kg\textquotesingle{}}\NormalTok{] }\OtherTok{\textless{}{-}} \StringTok{\textquotesingle{}weight.kg\textquotesingle{}}
\FunctionTok{colnames}\NormalTok{(imc2017)[}\FunctionTok{colnames}\NormalTok{(imc2017)}\SpecialCharTok{==}\StringTok{\textquotesingle{}Weight.kg\textquotesingle{}}\NormalTok{] }\OtherTok{\textless{}{-}} \StringTok{\textquotesingle{}weight.kg\textquotesingle{}}
\NormalTok{imc2016 }\OtherTok{=}\NormalTok{ imc2016[,}\FunctionTok{c}\NormalTok{(}\StringTok{\textquotesingle{}Sex\textquotesingle{}}\NormalTok{, }\StringTok{\textquotesingle{}height.m\textquotesingle{}}\NormalTok{, }\StringTok{\textquotesingle{}weight.kg\textquotesingle{}}\NormalTok{)]}
\NormalTok{imc2017 }\OtherTok{=}\NormalTok{ imc2017[,}\FunctionTok{c}\NormalTok{(}\StringTok{\textquotesingle{}Sex\textquotesingle{}}\NormalTok{, }\StringTok{\textquotesingle{}height.m\textquotesingle{}}\NormalTok{, }\StringTok{\textquotesingle{}weight.kg\textquotesingle{}}\NormalTok{)]}
\NormalTok{imc2016}\SpecialCharTok{$}\NormalTok{amostra }\OtherTok{\textless{}{-}} \StringTok{\textquotesingle{}2016\textquotesingle{}}
\NormalTok{imc2017}\SpecialCharTok{$}\NormalTok{amostra }\OtherTok{\textless{}{-}} \StringTok{\textquotesingle{}2017\textquotesingle{}}

\CommentTok{\# Calculando o IMC de cada população}
\NormalTok{imc2016}\SpecialCharTok{$}\NormalTok{imc }\OtherTok{\textless{}{-}}\NormalTok{ imc2016}\SpecialCharTok{$}\NormalTok{weight.kg}\SpecialCharTok{/}\NormalTok{(imc2016}\SpecialCharTok{$}\NormalTok{height.m}\SpecialCharTok{\^{}}\DecValTok{2}\NormalTok{)}
\NormalTok{imc2017}\SpecialCharTok{$}\NormalTok{imc }\OtherTok{\textless{}{-}}\NormalTok{ imc2017}\SpecialCharTok{$}\NormalTok{weight.kg}\SpecialCharTok{/}\NormalTok{(imc2017}\SpecialCharTok{$}\NormalTok{height.m}\SpecialCharTok{\^{}}\DecValTok{2}\NormalTok{)}

\CommentTok{\# União dos dataframes}
\NormalTok{imc }\OtherTok{\textless{}{-}} \FunctionTok{rbind}\NormalTok{(imc2016, imc2017)}

\CommentTok{\# Segregação de dados entre Homens e Mulheres}
\NormalTok{imc\_feminino }\OtherTok{\textless{}{-}}\NormalTok{ imc[imc}\SpecialCharTok{$}\NormalTok{Sex }\SpecialCharTok{==} \StringTok{\textquotesingle{}F\textquotesingle{}}\NormalTok{,]}
\NormalTok{imc\_masculino }\OtherTok{\textless{}{-}}\NormalTok{ imc[imc}\SpecialCharTok{$}\NormalTok{Sex }\SpecialCharTok{==} \StringTok{\textquotesingle{}M\textquotesingle{}}\NormalTok{,]}
\end{Highlighting}
\end{Shaded}

Em seguida, executamos o teste de Fligner-Killeen para verificar se a
premissa de que as variâncias são iguais se aplica no experimento.

\begin{Shaded}
\begin{Highlighting}[]
\FunctionTok{fligner.test}\NormalTok{(imc }\SpecialCharTok{\textasciitilde{}}\NormalTok{ Sex, }\AttributeTok{data =}\NormalTok{ imc) }
\end{Highlighting}
\end{Shaded}

\begin{verbatim}
## 
##  Fligner-Killeen test of homogeneity of variances
## 
## data:  imc by Sex
## Fligner-Killeen:med chi-squared = 1.0504, df = 1, p-value = 0.3054
\end{verbatim}

O p-valor resultante, de 0,3054, indica que não há evidências o
suficiente para rejeitar a hipótese nula, ou seja, corrobora com a
hipótese de que as variâncias dos dois grupos são iguais. Executamos,
então, o teste de Shapiro-Wilker para verificar a premissa de
normalidade das amostras.

\begin{Shaded}
\begin{Highlighting}[]
\FunctionTok{shapiro.test}\NormalTok{(imc}\SpecialCharTok{$}\NormalTok{imc[imc}\SpecialCharTok{$}\NormalTok{Sex }\SpecialCharTok{==} \StringTok{"F"}\NormalTok{]) }
\end{Highlighting}
\end{Shaded}

\begin{verbatim}
## 
##  Shapiro-Wilk normality test
## 
## data:  imc$imc[imc$Sex == "F"]
## W = 0.91991, p-value = 0.3179
\end{verbatim}

\begin{Shaded}
\begin{Highlighting}[]
\FunctionTok{shapiro.test}\NormalTok{(imc}\SpecialCharTok{$}\NormalTok{imc[imc}\SpecialCharTok{$}\NormalTok{Sex }\SpecialCharTok{==} \StringTok{"M"}\NormalTok{]) }
\end{Highlighting}
\end{Shaded}

\begin{verbatim}
## 
##  Shapiro-Wilk normality test
## 
## data:  imc$imc[imc$Sex == "M"]
## W = 0.94947, p-value = 0.0618
\end{verbatim}

Novamente, não há evidências para rejeitar \(H_0\). Com as premissas
validadas, executamos o t-test:

\begin{Shaded}
\begin{Highlighting}[]
\FunctionTok{t.test}\NormalTok{(imc}\SpecialCharTok{$}\NormalTok{imc }\SpecialCharTok{\textasciitilde{}}\NormalTok{ imc}\SpecialCharTok{$}\NormalTok{Sex, }
       \AttributeTok{alternative =} \StringTok{"less"}\NormalTok{, }
       \AttributeTok{mu          =} \DecValTok{0}\NormalTok{, }
       \AttributeTok{var.equal   =} \ConstantTok{TRUE}\NormalTok{, }
       \AttributeTok{conf.level  =} \FloatTok{0.95}\NormalTok{)}
\end{Highlighting}
\end{Shaded}

\begin{verbatim}
## 
##  Two Sample t-test
## 
## data:  imc$imc by imc$Sex
## t = -3.6409, df = 51, p-value = 0.0003175
## alternative hypothesis: true difference in means between group F and group M is less than 0
## 95 percent confidence interval:
##       -Inf -2.421576
## sample estimates:
## mean in group F mean in group M 
##        20.12522        24.61073
\end{verbatim}

Finalmente, identificamos o poder do teste

\begin{Shaded}
\begin{Highlighting}[]
\FunctionTok{power.t.test}\NormalTok{(}\AttributeTok{delta       =} \FloatTok{2.5}\NormalTok{, }\CommentTok{\# mínimo que faz a mudança de faixa de IMC}
             \AttributeTok{sd          =} \FunctionTok{sd}\NormalTok{(imc}\SpecialCharTok{$}\NormalTok{imc),}
             \AttributeTok{sig.level   =} \FloatTok{0.05}\NormalTok{,}
             \AttributeTok{n =} \FunctionTok{length}\NormalTok{(imc}\SpecialCharTok{$}\NormalTok{imc),}
             \AttributeTok{type        =} \StringTok{"two.sample"}\NormalTok{,}
             \AttributeTok{alternative =} \StringTok{"one.sided"}\NormalTok{)}
\end{Highlighting}
\end{Shaded}

\begin{verbatim}
## 
##      Two-sample t test power calculation 
## 
##               n = 53
##           delta = 2.5
##              sd = 4.043382
##       sig.level = 0.05
##           power = 0.935392
##     alternative = one.sided
## 
## NOTE: n is number in *each* group
\end{verbatim}

\subsubsection{Diferença de IMC entre os dois anos em
estudo;}\label{diferenuxe7a-de-imc-entre-os-dois-anos-em-estudo-1}

Para esse o próximo experimos utilizaremos os dados carregados na seção
anterior. Validamos as premissas de normalidade e variâncias da seguinte
forma:

\begin{Shaded}
\begin{Highlighting}[]
\FunctionTok{fligner.test}\NormalTok{(imc }\SpecialCharTok{\textasciitilde{}}\NormalTok{ amostra, }\AttributeTok{data =}\NormalTok{ imc\_feminino) }
\end{Highlighting}
\end{Shaded}

\begin{verbatim}
## 
##  Fligner-Killeen test of homogeneity of variances
## 
## data:  imc by amostra
## Fligner-Killeen:med chi-squared = 0.71101, df = 1, p-value = 0.3991
\end{verbatim}

\begin{Shaded}
\begin{Highlighting}[]
\FunctionTok{shapiro.test}\NormalTok{(imc\_feminino}\SpecialCharTok{$}\NormalTok{imc[imc\_feminino}\SpecialCharTok{$}\NormalTok{amostra }\SpecialCharTok{==} \StringTok{"2017"}\NormalTok{])}
\end{Highlighting}
\end{Shaded}

\begin{verbatim}
## 
##  Shapiro-Wilk normality test
## 
## data:  imc_feminino$imc[imc_feminino$amostra == "2017"]
## W = 0.7475, p-value = 0.03659
\end{verbatim}

\begin{Shaded}
\begin{Highlighting}[]
\FunctionTok{hist}\NormalTok{(imc\_feminino}\SpecialCharTok{$}\NormalTok{imc[imc\_feminino}\SpecialCharTok{$}\NormalTok{amostra }\SpecialCharTok{==} \StringTok{"2017"}\NormalTok{])}
\end{Highlighting}
\end{Shaded}

\includegraphics{report_template_files/figure-latex/normalidade e variancias iguais-1.pdf}

\begin{Shaded}
\begin{Highlighting}[]
\FunctionTok{qqnorm}\NormalTok{(imc\_feminino}\SpecialCharTok{$}\NormalTok{imc[imc\_feminino}\SpecialCharTok{$}\NormalTok{amostra }\SpecialCharTok{==} \StringTok{"2017"}\NormalTok{])}
\FunctionTok{qqline}\NormalTok{(imc\_feminino}\SpecialCharTok{$}\NormalTok{imc[imc\_feminino}\SpecialCharTok{$}\NormalTok{amostra }\SpecialCharTok{==} \StringTok{"2017"}\NormalTok{])}
\end{Highlighting}
\end{Shaded}

\includegraphics{report_template_files/figure-latex/normalidade e variancias iguais-2.pdf}

\begin{Shaded}
\begin{Highlighting}[]
\FunctionTok{shapiro.test}\NormalTok{(imc\_feminino}\SpecialCharTok{$}\NormalTok{imc[imc\_feminino}\SpecialCharTok{$}\NormalTok{amostra }\SpecialCharTok{==} \StringTok{"2016"}\NormalTok{]) }
\end{Highlighting}
\end{Shaded}

\begin{verbatim}
## 
##  Shapiro-Wilk normality test
## 
## data:  imc_feminino$imc[imc_feminino$amostra == "2016"]
## W = 0.91974, p-value = 0.4674
\end{verbatim}

\begin{Shaded}
\begin{Highlighting}[]
\FunctionTok{hist}\NormalTok{(imc\_feminino}\SpecialCharTok{$}\NormalTok{imc[imc\_feminino}\SpecialCharTok{$}\NormalTok{amostra }\SpecialCharTok{==} \StringTok{"2016"}\NormalTok{])}
\end{Highlighting}
\end{Shaded}

\includegraphics{report_template_files/figure-latex/normalidade e variancias iguais-3.pdf}

\begin{Shaded}
\begin{Highlighting}[]
\FunctionTok{qqnorm}\NormalTok{(imc\_feminino}\SpecialCharTok{$}\NormalTok{imc[imc\_feminino}\SpecialCharTok{$}\NormalTok{amostra }\SpecialCharTok{==} \StringTok{"2016"}\NormalTok{])}
\FunctionTok{qqline}\NormalTok{(imc\_feminino}\SpecialCharTok{$}\NormalTok{imc[imc\_feminino}\SpecialCharTok{$}\NormalTok{amostra }\SpecialCharTok{==} \StringTok{"2016"}\NormalTok{])}
\end{Highlighting}
\end{Shaded}

\includegraphics{report_template_files/figure-latex/normalidade e variancias iguais-4.pdf}

Experimento:

\begin{Shaded}
\begin{Highlighting}[]
\FunctionTok{t.test}\NormalTok{(imc\_feminino}\SpecialCharTok{$}\NormalTok{imc }\SpecialCharTok{\textasciitilde{}}\NormalTok{ imc\_feminino}\SpecialCharTok{$}\NormalTok{amostra, }
       \AttributeTok{alternative =} \StringTok{"less"}\NormalTok{, }
       \AttributeTok{mu          =} \DecValTok{0}\NormalTok{, }
       \AttributeTok{var.equal   =} \ConstantTok{TRUE}\NormalTok{, }
       \AttributeTok{conf.level  =} \FloatTok{0.95}\NormalTok{)}
\end{Highlighting}
\end{Shaded}

\begin{verbatim}
## 
##  Two Sample t-test
## 
## data:  imc_feminino$imc by imc_feminino$amostra
## t = 1.9308, df = 9, p-value = 0.9572
## alternative hypothesis: true difference in means between group 2016 and group 2017 is less than 0
## 95 percent confidence interval:
##     -Inf 5.14222
## sample estimates:
## mean in group 2016 mean in group 2017 
##           21.08443           18.44660
\end{verbatim}

\subsubsection{Diferença entre alunos de Graduação e Pós-Graduação em
2016;}\label{diferenuxe7a-entre-alunos-de-graduauxe7uxe3o-e-puxf3s-graduauxe7uxe3o-em-2016-1}

\begin{Shaded}
\begin{Highlighting}[]
\FunctionTok{fligner.test}\NormalTok{(imc }\SpecialCharTok{\textasciitilde{}}\NormalTok{ amostra, }\AttributeTok{data =}\NormalTok{ imc\_feminino) }\CommentTok{\# H0: VARIANCIAS IGUAIS}
\end{Highlighting}
\end{Shaded}

\begin{verbatim}
## 
##  Fligner-Killeen test of homogeneity of variances
## 
## data:  imc by amostra
## Fligner-Killeen:med chi-squared = 0.71101, df = 1, p-value = 0.3991
\end{verbatim}

\begin{Shaded}
\begin{Highlighting}[]
\FunctionTok{shapiro.test}\NormalTok{(imc\_masculino}\SpecialCharTok{$}\NormalTok{imc) }\CommentTok{\# p{-}value = 0.3179 {-}\textgreater{} aceitamos H0}
\end{Highlighting}
\end{Shaded}

\begin{verbatim}
## 
##  Shapiro-Wilk normality test
## 
## data:  imc_masculino$imc
## W = 0.94947, p-value = 0.0618
\end{verbatim}

\begin{Shaded}
\begin{Highlighting}[]
\FunctionTok{shapiro.test}\NormalTok{(imc\_masculino}\SpecialCharTok{$}\NormalTok{imc[imc\_masculino}\SpecialCharTok{$}\NormalTok{amostra }\SpecialCharTok{==} \StringTok{"2017"}\NormalTok{])}
\end{Highlighting}
\end{Shaded}

\begin{verbatim}
## 
##  Shapiro-Wilk normality test
## 
## data:  imc_masculino$imc[imc_masculino$amostra == "2017"]
## W = 0.96494, p-value = 0.6206
\end{verbatim}

\begin{Shaded}
\begin{Highlighting}[]
\FunctionTok{hist}\NormalTok{(imc\_masculino}\SpecialCharTok{$}\NormalTok{imc[imc\_masculino}\SpecialCharTok{$}\NormalTok{amostra }\SpecialCharTok{==} \StringTok{"2017"}\NormalTok{])}
\end{Highlighting}
\end{Shaded}

\includegraphics{report_template_files/figure-latex/teste entre turmas-1.pdf}

\begin{Shaded}
\begin{Highlighting}[]
\FunctionTok{qqnorm}\NormalTok{(imc\_masculino}\SpecialCharTok{$}\NormalTok{imc[imc\_masculino}\SpecialCharTok{$}\NormalTok{amostra }\SpecialCharTok{==} \StringTok{"2017"}\NormalTok{])}
\FunctionTok{qqline}\NormalTok{(imc\_masculino}\SpecialCharTok{$}\NormalTok{imc[imc\_masculino}\SpecialCharTok{$}\NormalTok{amostra }\SpecialCharTok{==} \StringTok{"2017"}\NormalTok{])}
\end{Highlighting}
\end{Shaded}

\includegraphics{report_template_files/figure-latex/teste entre turmas-2.pdf}

\begin{Shaded}
\begin{Highlighting}[]
\FunctionTok{shapiro.test}\NormalTok{(imc\_masculino}\SpecialCharTok{$}\NormalTok{imc[imc\_masculino}\SpecialCharTok{$}\NormalTok{amostra }\SpecialCharTok{==} \StringTok{"2016"}\NormalTok{]) }
\end{Highlighting}
\end{Shaded}

\begin{verbatim}
## 
##  Shapiro-Wilk normality test
## 
## data:  imc_masculino$imc[imc_masculino$amostra == "2016"]
## W = 0.92833, p-value = 0.1275
\end{verbatim}

\begin{Shaded}
\begin{Highlighting}[]
\FunctionTok{hist}\NormalTok{(imc\_masculino}\SpecialCharTok{$}\NormalTok{imc[imc\_masculino}\SpecialCharTok{$}\NormalTok{amostra }\SpecialCharTok{==} \StringTok{"2016"}\NormalTok{])}
\end{Highlighting}
\end{Shaded}

\includegraphics{report_template_files/figure-latex/teste entre turmas-3.pdf}

\begin{Shaded}
\begin{Highlighting}[]
\FunctionTok{qqnorm}\NormalTok{(imc\_masculino}\SpecialCharTok{$}\NormalTok{imc[imc\_masculino}\SpecialCharTok{$}\NormalTok{amostra }\SpecialCharTok{==} \StringTok{"2016"}\NormalTok{])}
\FunctionTok{qqline}\NormalTok{(imc\_masculino}\SpecialCharTok{$}\NormalTok{imc[imc\_masculino}\SpecialCharTok{$}\NormalTok{amostra }\SpecialCharTok{==} \StringTok{"2016"}\NormalTok{])}
\end{Highlighting}
\end{Shaded}

\includegraphics{report_template_files/figure-latex/teste entre turmas-4.pdf}

\begin{Shaded}
\begin{Highlighting}[]
\FunctionTok{t.test}\NormalTok{(imc\_masculino}\SpecialCharTok{$}\NormalTok{imc }\SpecialCharTok{\textasciitilde{}}\NormalTok{ imc\_masculino}\SpecialCharTok{$}\NormalTok{amostra, }
       \AttributeTok{alternative =} \StringTok{"less"}\NormalTok{, }
       \AttributeTok{mu          =} \DecValTok{0}\NormalTok{, }
       \AttributeTok{var.equal   =} \ConstantTok{TRUE}\NormalTok{, }
       \AttributeTok{conf.level  =} \FloatTok{0.95}\NormalTok{)}
\end{Highlighting}
\end{Shaded}

\begin{verbatim}
## 
##  Two Sample t-test
## 
## data:  imc_masculino$imc by imc_masculino$amostra
## t = 0.53979, df = 40, p-value = 0.7038
## alternative hypothesis: true difference in means between group 2016 and group 2017 is less than 0
## 95 percent confidence interval:
##      -Inf 2.679482
## sample estimates:
## mean in group 2016 mean in group 2017 
##           24.93595           24.28551
\end{verbatim}

\subsection{Discussões e
Conclusões}\label{discussuxf5es-e-conclusuxf5es}

Em todos os testes, o fato de trabalharmos com dados históricos
apresentou-se como uma complexidade que nos obrigou a ter uma cautela
adicional. O fato de possuirmos diferentes tamanhos de amostras em todos
os testes fez com que procuremos explorar o máximo o poder dos testes
utilizando por exemplo testes em que podíamos considerar variâncias
iguais.

O intuito de extrapolar os resultados de uma amostra centrada em uma
única disciplina para todo o departamento de Engenharia Elétrica rompe
com as premissas básicas de uma amostra iid, sendo que nos apoiamos na
suposição de que como qualquer aluno pode se matricular na disciplina os
alunos matriculados representariam todo o curso. Essa suposição, apesar
de ser razoável considerando que a altura e peso são fatores comumente
individuais, poderia não ser fidedigna com os testes, sendo que
possíveis estudos de correlação entre indivíduos poderia ser realizada
para fortalecer mais as conclusões estabelecidas.

\subsubsection{Diferença entre IMC de Homens e
Mulheres}\label{diferenuxe7a-entre-imc-de-homens-e-mulheres-2}

Corroborando a rejeição da hipótese nula, pode-se assumir que o IMC
feminino de fato é menor que o IMC masculino. Tais resultados podem
advir de diferenças na composição muscular e de gordura, assim como a
própria fisionomia. Maiores estudos podem ser realizados para validar
tais hipóteses desde que novos dados sejam extraidos.

\subsubsection{Diferença de IMC entre os dois anos em
estudo}\label{diferenuxe7a-de-imc-entre-os-dois-anos-em-estudo-2}

Ao compararmos o IMC dos dois anos de estudo, todas hipóteses de
normalidade do teste foram corroboradas. Além disso, não foi possível
rejeitar a hipótese nula, inclusive apresentando um p-valor
significativamente alto, sendo possível afirmar que não possuímos
certeza suficiente para considerar que os dois anos possuem diferenças
significativas na média do IMC.

\subsubsection{Diferença entre alunos de Graduação e Pós-Graduação em
2016}\label{diferenuxe7a-entre-alunos-de-graduauxe7uxe3o-e-puxf3s-graduauxe7uxe3o-em-2016-2}

Na realização deste experimento, é interessante notar que o teste de
normalidade não possuiu nível de significância estatística esperada,
apesar disto devido ao teste t ser robusto à variações de normalidade o
experimento pode ser executado. Novamente não podemos afirmar de forma
fidedigna que as duas amostras possuem diferença no valor das médias do
IMC, sendo que experimentos com um maior número de dados e maior
aleatoriedade poderiam fornecer maior significado estatístico.

\ldots{}

\begin{Shaded}
\begin{Highlighting}[]
\NormalTok{model}\OtherTok{\textless{}{-}}\FunctionTok{aov}\NormalTok{(mpg}\SpecialCharTok{\textasciitilde{}}\NormalTok{am}\SpecialCharTok{*}\NormalTok{disp,}\AttributeTok{data=}\NormalTok{mtcars)}
\FunctionTok{summary}\NormalTok{(model)}
\end{Highlighting}
\end{Shaded}

\begin{verbatim}
##             Df Sum Sq Mean Sq F value   Pr(>F)    
## am           1  405.2   405.2  47.948 1.58e-07 ***
## disp         1  420.6   420.6  49.778 1.13e-07 ***
## am:disp      1   63.7    63.7   7.537   0.0104 *  
## Residuals   28  236.6     8.4                     
## ---
## Signif. codes:  0 '***' 0.001 '**' 0.01 '*' 0.05 '.' 0.1 ' ' 1
\end{verbatim}

\subsubsection{Checking Model
Assumptions}\label{checking-model-assumptions}

The assumptions of your test should also be validated, and possible
effects of violations should also be explored.

\begin{Shaded}
\begin{Highlighting}[]
\FunctionTok{par}\NormalTok{(}\AttributeTok{mfrow=}\FunctionTok{c}\NormalTok{(}\DecValTok{2}\NormalTok{,}\DecValTok{2}\NormalTok{), }\AttributeTok{mai=}\NormalTok{.}\DecValTok{3}\SpecialCharTok{*}\FunctionTok{c}\NormalTok{(}\DecValTok{1}\NormalTok{,}\DecValTok{1}\NormalTok{,}\DecValTok{1}\NormalTok{,}\DecValTok{1}\NormalTok{))}
\FunctionTok{plot}\NormalTok{(model,}\AttributeTok{pch=}\DecValTok{16}\NormalTok{,}\AttributeTok{lty=}\DecValTok{1}\NormalTok{,}\AttributeTok{lwd=}\DecValTok{2}\NormalTok{)}
\end{Highlighting}
\end{Shaded}

\begin{figure}
\centering
\includegraphics{report_template_files/figure-latex/resplots-1.pdf}
\caption{Residual plots for the anova model}
\end{figure}

\subsubsection{Conclusions and
Recommendations}\label{conclusions-and-recommendations}

The discussion of your results, and the scientific/technical meaning of
the effects detected, should be placed here. Always be sure to tie your
results back to the original question of interest!

\phantomsection\label{refs}
\begin{CSLReferences}{0}{0}
\bibitem[\citeproctext]{ref-IMC_def}
\CSLLeftMargin{{[}1{]} }%
\CSLRightInline{Centers for Disease Control and Prevention, {``About
body mass index (BMI).''} 2024. Available:
\url{https://www.cdc.gov/bmi/about/index.html}}

\bibitem[\citeproctext]{ref-weir2019bmi}
\CSLLeftMargin{{[}2{]} }%
\CSLRightInline{C. B. Weir and A. Jan, {``BMI classification percentile
and cut off points,''} 2019.}

\bibitem[\citeproctext]{ref-nuttall2015body}
\CSLLeftMargin{{[}3{]} }%
\CSLRightInline{F. Q. Nuttall, {``Body mass index: Obesity, BMI, and
health: A critical review,''} \emph{Nutrition today}, vol. 50, no. 3,
pp. 117--128, 2015.}

\bibitem[\citeproctext]{ref-shah2012measuring}
\CSLLeftMargin{{[}4{]} }%
\CSLRightInline{N. R. Shah and E. R. Braverman, {``Measuring adiposity
in patients: The utility of body mass index (BMI), percent body fat, and
leptin,''} \emph{PloS one}, vol. 7, no. 4, p. e33308, 2012.}

\bibitem[\citeproctext]{ref-branco2023bioelectrical}
\CSLLeftMargin{{[}5{]} }%
\CSLRightInline{M. G. Branco \emph{et al.}, {``Bioelectrical impedance
analysis (BIA) for the assessment of body composition in oncology: A
scoping review,''} \emph{Nutrients}, vol. 15, no. 22, p. 4792, 2023.}

\bibitem[\citeproctext]{ref-bredella2017sex}
\CSLLeftMargin{{[}6{]} }%
\CSLRightInline{M. A. Bredella, {``Sex differences in body
composition,''} \emph{Sex and gender factors affecting metabolic
homeostasis, diabetes and obesity}, pp. 9--27, 2017.}

\bibitem[\citeproctext]{ref-olfert2018self}
\CSLLeftMargin{{[}7{]} }%
\CSLRightInline{M. D. Olfert \emph{et al.}, {``Self-reported vs.
Measured height, weight, and BMI in young adults,''} \emph{International
journal of environmental research and public health}, vol. 15, no. 10,
p. 2216, 2018.}

\bibitem[\citeproctext]{ref-graves2022undergraduate}
\CSLLeftMargin{{[}8{]} }%
\CSLRightInline{R. J. Graves \emph{et al.}, {``Undergraduate versus
graduate nursing students: Differences in nutrition, physical activity,
and self-reported body mass index,''} \emph{Journal of American College
Health}, vol. 70, no. 7, pp. 1941--1946, 2022.}

\end{CSLReferences}

\end{document}
